 %% Präambel
\documentclass[
	11homerhopt,
	a4paper,
	oneside, 
	liststotoc, 					% Tabellen- und Abbildungsverzeichnis ins Inhaltsverzeichnis
	bibtotoc,						  % Literaturverzeichnis ins Inhaltsverzeichnis aufnehmen
	titlepage, 						% Titlepage-Umgebung statt \maketitle
	headsepline, 					% horizontale Linie unter Kolumnentitel
]{scrreprt}


\usepackage[english]{babel} 		% deutsche Trennungsregeln und Übersetzung der festcodierten Überschriften
\usepackage[T1]{fontenc} 			% Ausgabe aller zeichen in einer T1-Codierung (wichtig für die Ausgabe von Umlauten!)
\usepackage{graphicx}  				% Einbinden von Grafiken erlauben
\usepackage{textcomp} 				% zum Einsatz von Eurozeichen u. a. Symbolen
\usepackage{listings}				% Datstellung von Quellcode mit den Umgebungen {lstlisting}, \lstinline und \lstinputlisting
\usepackage[dvipsnames, table]{xcolor} 			% einfache Verwendung von Farben in nahezu allen Farbmodellen
\usepackage[intoc]{nomencl} 		% zur Erstellung des Abkürzungsberzeichnisses
\makenomenclature					% Abkürzungsverzeichnis erstellen
\usepackage{setspace}				% Zusatzpaket zur Einstellung des Zeilenabstands
\usepackage{longtable}
 
\usepackage[lmargin=2.5cm, rmargin=2.5cm,tmargin=2.50cm,bmargin=2.50cm]{geometry}		% Seitenrändekr formatieren
\usepackage[style=alphabetic,citestyle=alphabetic,backend=bibtex]{biblatex}
\addbibresource[]{bib/bibtex.bib}

\usepackage{url}
\usepackage{eurosym}
\usepackage{amssymb}
\usepackage{amsmath}
\usepackage{amsthm}
\usepackage{todonotes}

% Fonts
%\usepackage{courier}
\usepackage[scaled]{beramono} % script font
\usepackage{caption}  % adapted captions
\usepackage{textcomp} % straight single quotess

% Check mark and cross mark
\usepackage{pifont}
\newcommand{\cmark}{\ding{51}}%
\newcommand{\xmark}{\ding{55}}%

% Table
\usepackage{booktabs}
\usepackage{arydshln}
\renewcommand{\arraystretch}{1.2}

\setlength{\nomlabelwidth}{.20\hsize}

% -----------------------------------------------------------------------------------------------------------------
% Zum Aktualisieren des Abkürzungsverzeichnisses bitte auf der Kommandozeile folgenden Befehl aufrufen :
%  makeindex Bachelorarbeit.nlo -s nomencl.ist -o Bachelorarbeit.nls
% -----------------------------------------------------------------------------------------------------------------

\definecolor{grey}{RGB}{050,050,050}
\definecolor{lightblue}{RGB}{224,238,238}
\definecolor{lightgrey}{RGB}{240,240,240}
\definecolor{darkgray}{rgb}{.4,.4,.4}
\definecolor{js_grey}{HTML}{778899}
\definecolor{js_blue}{HTML}{0077AA}
\definecolor{js_red}{HTML}{DD4A68}
\definecolor{js_green}{HTML}{669900}


\newcommand{\tab}{\qquad \qquad \qquad \qquad \qquad \qquad \qquad}
\setlength{\parindent}{0em}  	% Absatzeinrückung eliminieren

% ------------------------------ Persönliche Daten ------------------------------------------

\newcommand{\thesisTitle}{Application Security}
\newcommand{\thesisSubtitle}{Investigation and Validation of NoSQL Injection Vulnerabilities}
\newcommand{\work}{Master Thesis}
\newcommand{\studyCourse}{Computer Science}
\newcommand{\auth}{Patrick Spiegel}
\newcommand{\studendNumber}{1854488}
\newcommand{\deadline}{\today}
\newcommand{\period}{09. Mai - 08. November 2016}
\newcommand{\professor}{Prof. Dr. J\"orn M\"uller-Quade}
\newcommand{\professorTwo}{Prof. Dr. Dennis Hofheinz}
\newcommand{\companySupervisor}{Dr. Martin Johns}

% ------------------------------ Befehle ----------------------------------------------------
\newcommand{\jahr}{2016}			% für Angabe im Copyright-Vermerk der Titelseite

\newcommand{\changefont}[3]{\fontfamily{#1} \fontseries{#2} \fontshape{#3} \selectfont}

% ------------------------------ Abkürzungen ------------------------------------------------
\newcommand{\ua}{\mbox{u.\,a.\ }}
\newcommand{\zB}{\mbox{z.\,B.\ }}
\newcommand{\bs}{$\backslash$}

% ------------------------------ Titeländerungen --------------------------------------------
\renewcommand{\nomname}{List of Abbreviations}
\renewcommand{\lstlistlistingname}{List of Listings}

% -------------------------------------------------------------------------------------------
% Design von Listings
% -------------------------------------------------------------------------------------------
\lstset{
  frame=top,frame=bottom,
  basicstyle=\ttfamily\footnotesize,    % the size of the fonts that are used for the code
  stepnumber=1,                           % the step between two line-numbers. If it is 1 each line will be numbered
  numbersep=10pt,                         % how far the line-numbers are from the code
  tabsize=2,                              % tab size in blank spaces
  extendedchars=true,                     %
  breaklines=true,                        % sets automatic line breaking
  captionpos=t,                           % sets the caption-position to top
  stringstyle=\color{blue}\ttfamily, % Farbe der String
  showspaces=false,           % Leerzeichen anzeigen ?
  showtabs=false,             % Tabs anzeigen ?
  xleftmargin=17pt,
  framexleftmargin=17pt,
  framexrightmargin=17pt,
  framexbottommargin=2pt,
  framextopmargin=2pt,
  showstringspaces=false,
  numbers=left,
  numberstyle=\tiny,
  upquote=true
}

\DeclareCaptionFormat{listing}{\vskip1pt#1#2#3}
\captionsetup[lstlisting]{format=listing,singlelinecheck=false, margin=0pt, font={small,sf},labelsep=space,labelfont=bf}
\captionsetup[figure]{format=listing,singlelinecheck=false, margin=0pt, font={small,sf},labelsep=space,labelfont=bf, justification=centering}

\lstdefinelanguage{JavaScript}{
  keywords={typeof, new, true, false, catch, function, return, null, catch, switch, var, if, in, while, do, else, case, break},
  keywordstyle=\color{js_blue}\bfseries,
  ndkeywords={class, export, boolean, throw, implements, import, this, delete},
  ndkeywordstyle=\color{js_green}\bfseries,
  identifierstyle=\color{black},
  sensitive=false,
  comment=[l]{//},
  morecomment=[s]{/*}{*/},
  commentstyle=\color{js_grey}\ttfamily,
  stringstyle=\color{js_red}\ttfamily,
  morestring=[b]',
  morestring=[b]"
}

% -------------------------------------------------------------------------------------------
% Definition of Abbreviation-file 
% -------------------------------------------------------------------------------------------
\nomenclature{SQL}{Structured Query Language}
\nomenclature{NoSQL}{Not only Structured Query Language}
\nomenclature{HTTP}{Hypertext Transfer Protocol}
\nomenclature{DoS}{Denial of Service}
\nomenclature{CSRF}{Cross-site Request Forgery}
\nomenclature{PHP}{PHP Hypertext Preprocessor}
\nomenclature{URI}{Uniform Resource Identifier}
\nomenclature{SOAP}{Simple Object Access Protocol}
\nomenclature{WSDL}{Web Services Description Language}				% Datei mit Abkürzungen laden

% ------------------------------ Hyperref ------------------------------------------
\usepackage{hyperref}
\hypersetup{
    bookmarks=true,         % show bookmarks bar?
    unicode=false,          % non-Latin characters in Acrobat’s bookmarks
    pdftoolbar=true,        % show Acrobat’s toolbar?
    pdfmenubar=true,        % show Acrobat’s menu?
    pdffitwindow=false,     % window fit to page when opened
    pdfstartview={FitH},    % fits the width of the page to the window
    pdftitle={\thesisSubtitle}, % title
    pdfauthor={\auth},      % author
    pdfsubject={\work},     % subject of the document
    pdfcreator={\auth},     % creator of the document
    pdfproducer={\auth},    % producer of the document
    pdfkeywords={keyword1, key2, key3}, % list of keywords
    pdfnewwindow=true,      % links in new PDF window
    colorlinks=false,       % false: boxed links; true: colored links
    linkcolor=red,          % color of internal links (change box color with linkbordercolor)
    citecolor=green,        % color of links to bibliography
    filecolor=magenta,      % color of file links
    urlcolor=cyan           % color of external links
}

% -------------------------------------------------------------------------------------------
%                     Beginn des Dokumenteninhalts
% -------------------------------------------------------------------------------------------
\begin{document}
\setcounter{secnumdepth}{3}				% Nummerierungstiefe fürs Inhaltsverzeichnis
\setcounter{tocdepth}{3}					% Tiefe des Inhaltsverzeichnises
\changefont{cmr}{m}{n}						% Schriftart definieren

% ------------------------------ Titelei -----------------------------------------------------
\thispagestyle{plain}
\begin{titlepage}
\enlargethispage{4.0cm}
%\sffamily 								% Serifenlose Grundschrift für die Titelseite einstellen

\begin{minipage}[hbt]{10cm}
	\textbf{Institut f\"ur Theoretische Informatik} \\
  Arbeitsgruppe Kryptographie und Sicherheit \\

  \professor \\
  \professorTwo \\
\end{minipage}
\hfill
\begin{minipage}[hbt]{8cm}
  \includegraphics[scale=.8]{images/logo_kit.pdf}
\end{minipage} \\ [16ex]


\begin{center}
\huge{\textsc{\textbf{\thesisTitle}}}\\[3ex]
\LARGE{\textbf{\thesisSubtitle}}\\[10ex]
\textcolor{grey}{
	\LARGE{\textbf{\work\\[1ex]}}
	\Large{of \studyCourse}\\[1ex]
	by\\[2ex] \LARGE{\textbf{\auth}} 
} \\[18ex]


\end{center}

\begin{flushleft}

\begin{tabular}{ll}
Date:					& \tab \deadline \\ [1ex]
Period of processing:			& \tab \period   \\ [1ex]
Matriculation number: 			& \tab \studendNumber \\ [1ex]
\\
Primary Reviewer: & \tab \professor \\ [1ex]
Secondary Reviewer: & \tab \professorTwo \\ [1ex]
Company Supervisor:  & \tab \companySupervisor \\ [1ex]

\end{tabular} 



\end{flushleft}

\end{titlepage} 				% erzeugt die Titelseite

\pagenumbering{Roman}             % große, römische Seitenzahlen für Titelei
\begin{onehalfspace}
  \addchap{Statement of Authorship}
I declare that I have developed and written the enclosed thesis completely by myself, and have not used sources or means without declaration in the text.
\\[10ex]

\underline{\hspace{6cm}} \newline
Karlsruhe,  \today \\[4ex]


 				% Einbinden der eidestattlichen Erklärung
  \chapter*{Abstract} %*-mark to no appear in the contents listing
In the last decade many new challenges, such as big data, changed the way we build applications. The generation of emerging NoSQL databases provides a solution for these challenges, but do they provide security? Regarding injection, there exists a prevalent opinion: "We are not building queries from strings, so we do not have to worry about injection vulnerabilities!" \\

NoSQL databases keep spreading into a wide range of modern applications, but security aspects stay behind. The fact that the most serious class of attack - injections - are not taken seriously in regard to NoSQL calls for action. This thesis addresses this problem and gives an comprehensive insight into the sate of injection attacks against non-relational databases. Existing attack models have to be reconsidered due to the trend towards heterogeneous system landscapes. This work introduces two new attacker models emerging with NoSQL architectures. The most prevalent technology stacks including typical application and database layers are precisely examined for potential injection vulnerabilities. A detailed analysis of employed technologies reveals new ways to influence the semantics of database queries. The presented attacks are accompanied by multiple practical demonstrations. In a further step, the found attacks are broken down to the underlying conceptional problems. This attack classification facilitates the creation of suitable counter-measures. Beyond this, an approach for efficient NoSQL injection mitigation is introduced. The presented solution adds an layer of security by optional query parameter validation. With this approach, the flexibility required by non-relational data models is addressed and a seamless integration into to all investigated application platforms is achievable. As a whole, this thesis provides an deep insight into the subject of NoSQL injection and feasible mitigation approaches.   				% Einbinden des Abstracts
\end{onehalfspace}
\tableofcontents							% Erzeugen des Inhalsverzeichnisses
\listoffigures 								% Erzeugen des Abbildungsverzeichnisses 
\listoftables 								% Erzeugen des Tabellenverzeichnisses
\lstlistoflistings

\printnomenclature							% Erzeugen des Abkürzungsverzeichnisses
% pdflatex - pdflatex - cmd: makeindex filename.nlo  -s nomencl.ist -o filename.nls        - pdflatex
\pagebreak

% --------------------------------------------------------------------------------------------
%                    Inhalt der Bachelorarbeit
%---------------------------------------------------------------------------------------------
\pagenumbering{arabic}              % arabische Seitenzahlen für den Hauptteil
\begin{onehalfspace}                % Zeilenabstand einstellen

  \chapter{Introduction}
\label{cha:introduction}

\section{Motivation}

\section{Objective}

\section{Contributions}

The main contributions of this thesis about NoSQL application security are:
\begin{description}
\item [Injection Attacks] are not a database oder application layer specific problem. Therefore, this thesis reveals new injection vectors for multiple NoSQL databases in combination with common application layers. With this broad range of investigated databases, it is shown that NoSQL injection is a general issue.
\item [Attacker Models] do not always equal the existing SQL injection attacker model. Since NoSQL databases are created to handle huge amounts of unstrucuted data from several sources, restrictions are low. These circumstances may allow the insertion of user controlled data in the import process. On account of this, an additonal attacker model is introduced to cover these new conditions.
\item [Injection Prevention] is hard to ensure, since the data and therefore the requests assume various shapes. In order to reduce the risk of injection attacks, this thesis breaks down known injection vetors to the underlying design problem. On basis of this a new database driver design is presented, that is resitent to known injection attacks.
\end{description}

\section{Structure}

   \chapter{Technical Background}
\label{cha:technicalBackground}
This chapter outlines the essential technologies and techniques in the context of NoSQL injection. Thereby, the relevant fundamentals in the scope of this thesis are covered.

\section{Underlying Technologies}
Within this section, the prevalent protocols and paradigms for data exchange between systems are explained.

\subsection{HTTP}
The Hypertext Transfer Protocol (HTTP) represents an application-layer protocol for the transfer of connected multimedia documents \cite{Berners-Lee1996}. Initially, HTTP was created regarding web-server to web-browser communication, but today the protocol is applied for various purposes \cite{Fielding:1999}. As a stateless protocol following a client-server model, HTTP itself does not retain any state between requests. While the protocol is often applied as a part of the TCP/IP stack, it is also applicable with other reliable and error-checked protocols on the transport layer. A required header block and an optional body block build the basic structure of each request and response. Depending on the intended kind of operation, different methods for requests exist. Table \ref{tab:http_methods} gives an overview of the available HTTP methods and their characteristics \cite{Fielding:1999}.\\

\begin{table}[h]
 \sffamily
 \centering
 \begin{tabular}{llccc}
  \textbf{\begin{tabular}{@{}c@{}}HTTP \\ Method\end{tabular}} & \textbf{Description} & \textbf{\begin{tabular}{@{}c@{}}Request \\ Body\end{tabular}} & \textbf{\begin{tabular}{@{}c@{}}Response \\ Body\end{tabular}} & \textbf{Idempotent} \\ \hline
  GET     & Retrieve specified resource & \xmark & \cmark & \cmark \\
  HEAD    & Retrieve specified resource headers & \xmark & \xmark & \cmark \\
  POST    & Create specified resource & \cmark & \cmark & \xmark \\
  PATCH   & Partially updated specified resource & \cmark & \cmark & \xmark \\
  PUT     & Replace specified resource & \cmark & \cmark & \cmark \\
  DELETE  & Delete specified resource & \xmark & \cmark & \cmark \\ \hdashline
  CONNECT & Convert connection to tunnel & \cmark & \cmark & \xmark \\
  OPTIONS & Return supported HTTP methods & \cmark & \cmark & \cmark \\
  TRACE   & Echo received request & \xmark & \cmark & \cmark \\
  \bottomrule
 \end{tabular}
 \caption{Overview of HTTP methods' characteristics}
 \label{tab:http_methods}
\end{table}

Basically, the first six methods implement create, read, update and delete operations for resources on the server. Thereby, a URI in the header specifies the resource, that the operation should be applied to. In case of create or update operations, the optional body block in the request contains the new resource data. Since no new data is needed for resource retrieving and deleting methods, they work solely with the request header block. All operations, apart form the HEAD request, deliver the result of the operation within the response's body block. Idempotents indicates, whether the behavior of a method stays the same for multiple consecutive requests. Especially update operations do not feature idempotent characteristics. The last three methods represent helper functions for connection inspection and TCP tunneling. An example for a request, that retrieves a specified resource form the server, is given in listing \ref{lst:http_request} \cite{MDN:2016}. \\

\begin{lstlisting}[caption={Example for HTTP GET request}, label={lst:http_request}]
GET /index.html HTTP/1.1
Host: example.org
Accept-Language: en
\end{lstlisting}

The first line of the request header is composed of the HTTP method, followed by resource identifier and completed with the version of the used protocol. In the given example a  GET request is used in order to retrieve the \textit{index.html} file. Afterwards, the host of the server is defined by the domain name. Accepted types of files, char-sets, encodings, languages and many more settings can be defined as additional headers. In case the requested resource exists and no error appears during request processing, the server sends a response as exemplified in listing \ref{lst:http_response} \cite{MDN:2016}. \\

\begin{lstlisting}[caption={Example for HTTP GET response}, label={lst:http_response}]
HTTP/1.1 200 OK
Date: Sat, 09 Sep 2016 14:28:02 GMT
Server: Apache
Last-Modified: Tue, 01 Dec 2009 20:18:22 GMT
ETag: "51142bc1-7449-479b075b2891b"
Accept-Ranges: bytes
Content-Length: 29769
Content-Type: text/html

<!DOCTYPE html... (here comes the 29769 bytes of the requested web page)
\end{lstlisting}

In the first line of the response, the server version of the protocol as well as the status code and message are returned. Depending on the success of the request, the status code and message in the response change. Further headers contain meta information about the server, the requested resource and the body block. Separated by a blank line, the body block follows the header. In this example the body contains the requested HTML document. With the end of the response body, the HTTP request-response cycle is finished.

\subsection{REST}
Representational State Transfer (REST) describes a programming paradigm for distributed systems in a stateless client-server environment \cite{Fielding:2000}. It specifies an architectural style, that enforces a consistent system interface design. Therefore, REST represents an alternative to prior approaches, such as SOAP or WDSL. The paradigm separates resource location and the applied method. Functional instructions are therefore not allowed to be included within the URI. A special implementation or protocol is not defined by the paradigm, whereby REST is commonly used in combination with HTTP and HTTPS. In the course of this, the HTTP methods encode the functional instructions and resources has to be uniquely identified by the requested URI. Though, the usage of HTTP and HTTPS does not imply REST conformance. Methods are often misused for different functionalities and as a result REST conformance is not provided.

\section{NoSQL Databases}
This section gives an introduction to NoSQL databases and the idea behind this new class of data storage. NoSQL databases and their concepts, that are relevant in scope of this work, are outlined in detail.

\subsection{Overview}
In 1998 the word \textit{NoSQL} was first used by Carlo Strozzi in context of a relational, document-oriented database designed to be accessed without SQL \cite{Strozzi:2007}. More than 10 years later in 2009, Johan Oskarsson reintroduced the term \textit{NoSQL} in oder to find a collective name for the increasing number of distributed, non-relational databases \cite{Oskarsson:2009}. From this time forward, NoSQL was used with the meaning of \textit{not only SQL} or \textit{non relational} instead of \textit{no SQL}. Therefore, the expression builds an umbrella term for databases without relational restrictions and often missing ACID support. The idea for such databases arose back in the 1960s \cite{IBM:2016, Nelson:1965}, but the actual trend started with the challenges of big data. NoSQL databases addressed these challenges through their flexible data models, distributed storage, horizontal scaling abilities and independence from dedicated hardware. Currently around 225 databases are prevalent, that denote themselves as NoSQL storage \cite{Edlich:2016}. Despite the shared non-relational concept, NoSQL databases can be divided in different categories. Depending on the focused field of application, there exist four major groups:

\begin{description}
\item [Key-value Stores] These databases reference datasets with unique keys. Datasets are accessed with the key, that retrieves a pointer to the dataset from a hash table. Therefore data access over the key is very fast, but querying or updating single properties of a dataset is inefficient. Notable examples for key-value stores are Redis, Riak, Voledmort or Memcached.
\item [Document Stores] The main idea here is similar to key-value stores, but datasets also are composed of semistructured key-value pairs. These datasets are often called documents and are persisted in formats like JSON. This allows more efficient querying for single properties of a dataset. Notable examples for document stores are MongoDB, CouchDB or CouchBase.
\item [Graph databases] This group uses a flexible structure with nodes and edges to model graph structures. Depending on the implementation and the underlying data persistence, the data access differs. Some databases implement RESTful interfaces, others provide dedicated query APIs. Notable examples for graph databases are Neo4j, Titan or Graphite.
\item [Column Stores] Similar to relational databases, data is very structured, but the underlying persistence layer differs strongly. Data is arranged column after column instead of row after row. This column oriented storage allows the processing and especially analysis of large amounts of data over distributed systems. Notable examples for column stores are Apache Cassandra, HBase or SAP HANA.
\end{description}

Apart from the presented categories of NoSQL databases, there also exist hybrid models implementing features originating from multiple categories. An example is OrientDB, that combines a document store with graph database abilities. 

\subsection{MongoDB}
The non-relational, document-oriented database MongoDB was published as an open-source project in 2009 \cite{Eliot:2010}. Deriving its name from the word humongous, it is designed to provide high performance, high availability as well as automatic horizontal scaling \cite{MongoDB_Intro:2016}. The JSON style data model enables complex hierarchies for stored documents, but still allows efficient querying and indexing. Documents are stored in Binary JSON (BSON) format into collections, that in turn can be queried by JSON expressions. BSON extends JSON with additional data types, such as date, time stamps and regular expressions and is binary encoded. By default every MongoDB instance holds special system collections for indexes, namespaces, users and JavaScript code. MongoDB provides official drivers for various popular programming languages next to many unofficial community-supported drivers. 

\subsection{Redis}
Redis is an in-memory operating key-value store for strings, hashes, lists and sets \cite{Sanfilippo:2016}. The 2009 released open-source database allows range queries and has built in Lua support for scripting. Due to its low data structure complexity, Redis provides large performance advantages in comparison to relational databases. Snapshotting or append only files allow on-disk persistence of the in-memory stored data. Later versions also allow distributed storage across clusters. Redis provides driver bindings for the most prevalent programming languages.

\subsection{Memcached}
The in 2003 released database Memcached is an in-memory key-value store for data caching \cite{Dormando:2015}. It allows to store and retrieve data from memory with hash tables distributed across multiple systems. Therefore, the result of requests to external sources can be cached and accessed much faster for subsequent access. Since the database is designed as a cache, data persistence on disc is not provided by default. Due to the resulting performance benefits of this approach, Memcached achieved wide prevalence and is deployed by companies like Facebook, YouTube, Twitter and Wikipedia. Drivers are provided for all major programming languages.


\subsection{CouchDB}
CouchDB is an document-oriented database written in Erlang \cite{Anderson:2010}. First published in 2005, the database is now maintained by the Apache Software Foundation and distributed under the Apache license. CouchDB is designed to combine the relaxed data model of document stores as well as the performance and scalability of relational databases. Therefore, the database offers horizontal scaling and implements the MapReduce framework for data processing. On Account of this, CouchDB includes Mozilla's JavaScript engine SpiderMonkey to facilitate scripting. Access for all major programming languages is provides with a RESTful HTTP interface. 

\section{Application Security}
In this section an outline of the relevant security topics of this thesis is given. Therefore, general security goals are introduced ,the current state of NoSQl security is blueprinted and afterwards, the concept of injection attacks is described.

\subsection{Security Goals}
Information security is a general term comprising the practice of data protection by defending it against attacks \cite{olivier2002database,Perrin:2008}. It involves three major security goals, that have to be assured for the underlying data. These goals describe a security model, known as the CIA triad. This fundamental concept of information security is defined as follows:

\begin{description}
\item [Confidentiality] Data has to be protected against any kind of unauthorized disclosure and the authenticity of data sources has to be ensured.
\item [Integrity] Data hast to be protected from any kind of unauthorized modification and its consistency, accuracy as well as trustworthy have to be ensured.
\item [Availability] Data hast to be accessible for authorized entities and usability as well as timely access hast to be ensured. A delay or denial of access for authorized entities or unauthorized removal of data have to be prevented.
\end{description}

Vulnerabilities like missing encryption, software bugs, weak passwords or inaccurate checks enable an attacker to break one of these goals for the targeted data. Attacks in turn are the act of exploiting vulnerabilities. Most security measures are designed in order to prevent suchlike attacks.

\subsection{NoSQL Security}
Non-relational databases represent a relatively new technological approach, thus security aspects did not attain perfection by now \cite{Okman:2011, Noiumkar:2014}. In order to give an overview of these security aspects, multiple facets of the current state of NoSQL security have to be considered. Non-volatile data persistence is an essential subject for databases. The data persisted on disc has to be protected from illegitimate disclosure from other entities. Unfortunately, many non-relational databases store data unencrypted. This situation leverages direct data access from disc, when the underlying system or one of its applications is compromised. Other vital features of a database are authentication and authorization checks. By default, instances of many NoSQL databases run without these mechanisms. This circumstance allows uncontrolled access on the stored data. When the database ports are exposed to the outside, attackers are able to run remote attacks. Closing these ports to the outside still leaves room for CSRF attacks, thanks to HTTP-based interfaces. Distributed storage is an major concern for non-relational databases and therefore inter-cluster communication becomes necessary. Some of the new databases are missing encryption for connections between distributed instances, enabling the eavesdropping of transfered data. Last of all, also injection attacks represent an important security aspect. Despite other query languages than SQL are used, the concatenation of string parameters still reveals attack surface. However, injection attacks against NoSQL databases constitute a relatively new and unexplored issue entailing a broad spectrum for detailed research.

\subsection{Injection Attacks}
The class of injection attacks encompasses a variety of actual attacks directed against specific contexts \cite{OWASP:2013}. These types of attacks are ranked as the number one web application security risk according to the OWASP top ten project \cite{OWASP:2013b}. An injection vulnerability allows manipulation of later-on interpreted contents. Suchlike vulnerabilities occur, when data originating from untrusted sources is passed unsanitized to an interpreter. Untrusted sources are user controlled and therefore offer attack surface for manipulation. In order to conduct an injection attack, the attacker crafts a text-based payload and passes it to the system. This payload exploits the syntax of the targeted interpreter to achieve unintended behavior. As a result, manipulated contents originating from the untrusted source are interpreted and executed. Depending on the attacked interpreted context, such attacks lead to data corruption, data loss, denial of access or even hostile system takeover. Common contexts targeted by injection attacks are SQL, Shell, LDAP or XML. Best practice for injection mitigation is the deployment of parameterized interfaces or alternatively stored procedures. The OWASP also recommends a strict input validation by white-listing in feasible cases.

  \chapter{Introduction to NoSQL Injection}

\section{From SQL to NoSQL Injection}

\section{Definition of NoSQL Injection}
\subsection{Common Attacker Model}

\begin{figure}[h]
\centering
  \includegraphics[width=1\linewidth]{Images/attacker_model_normal}
  \caption{Common attacker model for NoSQL injection}
  \label{fig:normalAttackerModel}
\end{figure}

% Objective

\subsection{Extended Attacker Model}

\begin{figure}[h]
\centering
  \includegraphics[width=1\linewidth]{Images/attacker_model_extended}
  \caption{Extended attacker model for NoSQL injection}
  \label{fig:extendedAttackerModel}
\end{figure}

% Objective

\subsection{Direct Attacker Model}

\begin{figure}[h]
\centering
  \includegraphics[width=1\linewidth]{Images/attacker_model_direct}
  \caption{Direct attacker model for NoSQL injection}
  \label{fig:extendedAttackerModel}
\end{figure}

% Objective


\section{Considered Technology Stack}
\subsection{Selected Databases}
\subsection{Selected Application Platforms}


\chapter{NoSQL Injection Attacks}

\section{MongoDB Document Store}
\subsection{Node.js Type Injection Attack}
\subsection{PHP Type Injection Attack}

\begin{lstlisting}[caption={Vulnerable PHP - MongoDb application}, label={lst:PHPArrayInjection}]
$collection->find(array('user' => $_GET['user'], 'password' => $_GET['password']));
\end{lstlisting}


\begin{lstlisting}[caption={MongoDB injection with PHP's associative arrays}, label={lst:PHPArrayInjection}]
https://example.org?user=patrick&password[%24ne]=1
\end{lstlisting}

\begin{lstlisting}[caption={Injected query parameter for MongoDB PHP injection}, label={lst:PHPArrayParam}]
{'user': 'patrick', 'password': {'&ne':1}}
\end{lstlisting}


\section{Redis Key-Value Store}
\subsection{Node.js Type Injection Attack}
\subsection{PHP Type Injection Attack}

\section{CouchDB Document Store}
\subsection{MapReduce Attack}
  \chapter{NoSQL Injection Attacks}
This chapter gives an overview of the in scope of this thesis found NoSQL injection attacks. Each of the presented attacks is accompanied by a vulnerable code example as well as a suitable attack vector. With regard to the attacker surface parameters analyzed in section \ref{sec:analysisOfQueryTechniques}, the vectors focus on query string and JSON encoded payloads. At this point it needs to be told, that the presented code snippets are created for demonstration purposes and should not be deployed in practice. The chapter is structured in one section per investigated database.

\section{MongoDB}
Within this section, the newly discovered injection attacks against the document store MongoDB are outlined.

\subsection{Query Selector Injection}
This attack is based on the common attacker model and is directed against the selection parameter of queries. MongoDB employs BSON, an extended JSON format, to define the selected documents affected by the query. Complex selection operations, such as \emph{greater than} or \emph{not equal}, are implemented with the help of a set of special query selectors. These are formatted as objects and replace the actual value within the query criteria as demonstrated in listing \ref{lst:MongoQuerySelectors}.  \\

\begin{lstlisting}[caption={Example for MongoDB's query selectors}, label={lst:MongoQuerySelectors}]
find({'password' : 'secure'});      // Password property equals "secure"
find({'password' : {'$ne': '0'}});  // Password property not equals "0"
\end{lstlisting}

The injection attack presented by Sullivan \cite{Sullivan:2011} showed, that query selectors in combination with PHP applications can be utilized for injection attacks. Listing \ref{lst:PHPQuerySelectorInjection} shows the according vulnerable code of an login implementation. \\

\begin{lstlisting}[caption={Vulnerable PHP example for query selector injection on MongoDB}, label={lst:PHPQuerySelectorInjection}]
$collection->find(array('user' => $_GET['user'], 'password' => $_GET['password']));
\end{lstlisting}

The code checks for documents, that match the passed user and password values from the query string. Here comes the automatic query string parsing of PHP into play. This feature enables the injection of a Query Selector instead of an actual password with the help of the extended query string syntax. An passed selector like \emph{no equals} will nearly always evaluate to true. With the manipulated request shown in \ref{lst:QuerySelectorInjectionVector}, the login check can be reliably bypassed. \\

\begin{lstlisting}[caption={Attack vector on MongoDB for query selector injection via the query string parameter}, label={lst:QuerySelectorInjectionVector}]
https://example.org/login?user=patrick&password[%24gt]=
\end{lstlisting}

Pektov \cite{Petkov:2014a} demonstrated a vulnerable login implementation running on NodeJS's Express web server following the same approach. The application code given in listing \ref{lst:NodeQuerySelectorInjection} can also be attacked with the request presented in listing \ref{lst:QuerySelectorInjectionVector} . \\

\begin{lstlisting}[caption={Vulnerable NodeJS example for query selector injection on MongoDB}, label={lst:NodeQuerySelectorInjection}]
db.collection('users').find({"user": req.query.user, "password": req.query.password});
\end{lstlisting}

So are there even any new findings for this class of attack? The previous publications presented the query selector injection as a distinct issue occurring only for PHP and Node applications, since its based a non-standardized application layer behavior. As explained in section \ref{sec:analysisOfQueryTechniques}, the query string parsing is a convenience functionality present across all investigated application platforms. The examples given in listing \ref{lst:RubyQuerySelectorInjection} for Ruby as well as listing \ref{lst:PythonQuerySelectorInjection} reveal, that query selector injection is an issue appearing across platforms. \\

\begin{lstlisting}[caption={Vulnerable Ruby example for query selector injection on MongoDB}, label={lst:RubyQuerySelectorInjection}]
db['users'].find({:user => req.params['user'], :password => req.params['password']})
\end{lstlisting}

\begin{lstlisting}[caption={Vulnerable Python example for query selector injection on MongoDB}, label={lst:PythonQuerySelectorInjection}]
db.users.find({"user": request.GET['user'], "password": request.GET['password']})
\end{lstlisting}

In all of the four cases, the default and by MongoDB recommended database driver was deployed. The attack vector from listing \ref{lst:QuerySelectorInjectionVector} leads in each of the presented login checks to the query criteria shown in listing \ref{lst:QuerySelectorCriteria}. \\

\begin{lstlisting}[caption={Resulting query of query selector injection}, label={lst:QuerySelectorCriteria}]
{'user': 'patrick', 'password': {'&gt':''}}
\end{lstlisting}

A login can therefore be bypassed in each investigated case, but this kind of attack is not restricted to login applications. Nearly all parameterized selection criteria can be influenced by the injection of query selectors. This includes all read, update and delete calls making use of the BSON selection format. Affected functions of the MongoDB API are listed in table \ref{tab:mongo_commands_affected}.\\


\begin{table}[h]
 \sffamily
 \centering
 \begin{tabular}{lll}
  \textbf{Command} & \textbf{Arguments} \\ \hline
  db.collection.deleteOne         & \textbf{filter} options \\
  db.collection.deleteMany        & \textbf{filter} options \\
  db.collection.find              & \textbf{query} projection \\
  db.collection.findAndModify     & \textbf{query} options \\
  db.collection.findOne           & \textbf{query} projection \\
  db.collection.findOneAndDelete  & \textbf{filter} options \\
  db.collection.findOneAndReplace & \textbf{filter} replacement options \\
  db.collection.findOneAndUpdate  & \textbf{filter} update options \\
  db.collection.replaceOne        & \textbf{filter} replacement options \\
  db.collection.remove            & \textbf{query} option \\
  db.collection.update            & \textbf{query} update options \\
  db.collection.updateOne         & \textbf{filter} update options \\
  db.collection.updateMany        & \textbf{filter} update options \\
  \bottomrule 
 \end{tabular}
 \caption{MongoDB commands affected by query selector injection}
 \label{tab:mongo_commands_affected}
\end{table}

Considered are collection functions, that employ ether a \emph{filter} or \emph{query} parameter. Both facilitate the same JSON-like selection format. The affected functions encompass read, update and delete operations. Furthermore, the attacker surface analysis from section \ref{sec:analysisOfQueryTechniques} suggests, that the URL query string is by far not the only part of the attacker surface allowing this kind of attack. When the application processes other request parameters, the payload can also be placed in form data encoded as well as JSON request bodies. These circumstances exhibit a way more significant surface for query selector injection as previously indicated. \\

\subsection{Expanding Array Injection}
The expanding array injection is based on the common attacker model and exploits a special behavior of a query's selection criteria. When a requires a property to be equal to a value and the stored document contains an array in place of the property, all values of the array are used to find a match. This feature, as the attack name implies, is called automatic array expanding. An attacker can manipulate inserted documents in order to achieve matches subsequent selective criteria. The creation of an user as shown in listing \ref{lst:NodeJSCreateUser} represents an example for such an vulnerability. \\

\begin{lstlisting}[caption={Example for vulnerable MongoDB - NodeJS application}, label={lst:NodeJSCreateUser}]
if (req.query.user !== "admin") {
  db.collection('users').insert({"user": req.query.user, "password": req.query.password});
}
\end{lstlisting}

The given code allows the insertion of new users into the database, with the exception of \emph{admin} as a username. This security check can be bypassed, by injecting an array of usernames instead of a simple string. A suitable attack vector based on query string encoding is given with listing \ref{lst:ExpandingArrayInjection}.\\

\begin{lstlisting}[caption={MongoDB injection with NodeJS's query string module}, label={lst:ExpandingArrayInjection}]
https://example.org/register?user[]=patrick&user[]=admin&password=1234;
\end{lstlisting}

The given request creates an array-typed \emph{user} property that leads to the bypass of the if-clause. In the next step, the query is executed and the provided data results into the document presented in listing \ref{lst:ExpandingArrayInjectionDocument} stored within the user collection.\\

\begin{lstlisting}[caption={Injected query parameter for MongoDB - NodeJS injection}, label={lst:ExpandingArrayInjectionDocument}]
{'user': ['patrick', 'admin'], 'password': 1234}
\end{lstlisting}

At this point, the array expanding functionality becomes important. An attacker can now use the normal authentication with username and password as shown in listing \ref{lst:NodeJSArrayExpandingLogin} to login. This works for both of the users inserted with the initial request. \\

\begin{lstlisting}[caption={Example for vulnerable MongoDB - NodeJS application}, label={lst:NodeJSArrayExpandingLogin}]
db.collection('users').find({"user": req.query.user, "password": req.query.password});
\end{lstlisting}

The request for the login, does have to be manipulated. An attacker can now login as an amdin with a usual authentication request as shown in listing {lst:ArrayExpandingAdminLogin}.\\

\begin{lstlisting}[caption={MongoDB injection with NodeJS's query string module}, label={lst:ArrayExpandingAdminLogin}]
https://example.org/login?user=admin&password=1234;
\end{lstlisting}

MongoDb will search for a document with the user property set to \emph{admin} and the password property set to \emph{1234}. Through the expansion the the beforehand injected document or respectively the user property, a matching document is found and the admin login is successful.\\

The presented login application is of course not the only example, where the expanding array feature of MongoDB allow injection attacks. Every insert operation with data originating from the three parameters, that allow type manipulation represents a potential vulnerability. It enables an attacker to break the confidentiality of the underlying data, which leads to unintended matches for further requests. The attack was exemplified for the NodeJS, but is relevant for all of the investigated application layers. \\

\begin{table}[h]
 \sffamily
 \centering
 \begin{tabular}{lllll}
  \textbf{Command} & \textbf{Arguments} \\ \hline
  db.collection.insert            & \multicolumn{3}{l}{\textcolor{dark-red}{\textbf{document}} | [ \textcolor{dark-red}{\textbf{document ... }} ] } \\
  db.collection.insertOne         & \textcolor{dark-red}{\textbf{document}} \\ 
  db.collection.update            & \textcolor{dark-blue}{\textbf{query}} \textcolor{dark-red}{\textbf{update}} options \\
  db.collection.updateOne         & \textcolor{dark-blue}{\textbf{filter}} \textcolor{dark-red}{\textbf{update}} options \\
  db.collection.updateMany        & \textcolor{dark-blue}{\textbf{filter}} \textcolor{dark-red}{\textbf{update}} options \\
  db.collection.replaceOne        & \textcolor{dark-blue}{\textbf{filter}} \textcolor{dark-red}{\textbf{replacement}} options \\\hline
  db.collection.deleteOne         & \textcolor{dark-blue}{\textbf{filter}} options \\
  db.collection.deleteMany        & \textcolor{dark-blue}{\textbf{filter}} options \\
  db.collection.find              & \textcolor{dark-blue}{\textbf{query}} projection \\
  db.collection.findAndModify     & \textcolor{dark-blue}{\textbf{query}} options \\
  db.collection.findOne           & \textcolor{dark-blue}{\textbf{query}} projection \\
  db.collection.findOneAndDelete  & \textcolor{dark-blue}{\textbf{filter}} options \\
  db.collection.findOneAndReplace & \textcolor{dark-blue}{\textbf{filter}} replacement options \\
  db.collection.findOneAndUpdate  & \textcolor{dark-blue}{\textbf{filter}} update options \\
  db.collection.remove            & \textcolor{dark-blue}{\textbf{query}}  option \\
  \bottomrule 
 \end{tabular}
 \caption{MongoDB commands affected by expanding array injection}
 \label{tab:mongo_commands_affected_expanding_array}
\end{table}

The presented login application is of course not the only example, where the expanding array feature of MongoDB allow injection attacks. Every insert operation with data originating from the three parameters, that allow type manipulation represents a potential vulnerability. It enables an attacker to break the confidentiality of the underlying data, which leads to unintended matches for further requests. The attack was exemplified for the NodeJS, but is relevant for all of the investigated application layers. \\

\section{Redis}
This section covers the injection attacks against the key-value store Redis, that were found in scope of this thesis.

\subsection{Parameter Overwrite Injection}
The parameter overwrite injection against Redis is based on the common attacker model. It requires NodeJS as an application platform in combination with the most used Redis driver. In order to understand this attack, it is essential to know how parameters can be accessed to the this particular database driver. The regular way allows to pass each parameter as distinct argument. An alternative way allows to pass all parameters within a single array as the first argument of the function call. With this knowledge and the control of the first argument, an attacker is able to overwrite the following arguments of an driver function call. An example application, where this becomes critical is shown in listing \ref{lst:parameterOverwriteApp}. \\

\begin{lstlisting}[caption={Vulnerable NodeJS example for parameter overwrite injection on Redis}, label={lst:parameterOverwriteApp}]
RedisClient.expireat(req.query.key, new Date("November 8, 2026 11:13:00").getTime());
\end{lstlisting}

This code makes use of the \emph{expireat} functionality to set the data for the deletion of an entry. In this particular case, the expire date of the entry behind the passed key is set to a date far in the future. The user can therefore only extend the life of an entry behind an arbitrary key, but not really influence the stored data in a negative way. This situation changes, when the second parameter can be overwritten. A conceivable attack could set the expire data to a time in the past. An request, that triggers exactly this operation is shown in listing \ref{lst:parameterOverwriteAtt}. \\

\begin{lstlisting}[caption={Attack vector on Redis for query selector injection via HTTP GET}, label={lst:parameterOverwriteAtt}]
https://example.org/expire?key[]=foo&key[]=1117542887
\end{lstlisting}

Given the the query string parameters, the resulting array will be passed as the first argument to the database call. The following parameter, containing the future time will be overwritten. Instead the injected timestamp passed with with the array in the first argument will be used. Since this timestamp lays in the past, the stored data behind the provided key will be deleted. This enables an attacker to delete arbitrary entries of the database! Similar to the presented attack on \emph{expireat}, there exist multiple other affected database commands. An excerpt of commands, that are vulnerable for parameter overwrite injection are listed within table \ref{tab:redis_commands_affected}. \\

\begin{table}[h]
 \sffamily
 \centering
 \begin{tabular}{lll}
  \textbf{Command} & \textbf{Arguments} & \textbf{Injection Description} \\ \hline
  Append  & key value       & Append a value to an arbitrary key\\
  DecrBy  & key decrement   & Decrement the integer value of a selected key \\
  Del     & key [key ...]   & Delete multiple selected keys at once \\
  Exists  & key [key ...]   & Check existence of multiple keys at once \\
  Expire  & key second      & Delete arbitrary key \\
  ExpireAt& key timestamp & Delete arbitrary key \\
  GetRange& key start end & Get entire content of key \\
  GetSet  & key value & Overwrite value of selected key \\
  IncrBy  & key increment & Increment the integer value of a selected key \\
  Move    & key db & Set destination database \\
  Rename  & key newkey & Set new key \\
  Set     & key value [expire] & Set arbitrary key-value data \\
  \bottomrule 
 \end{tabular}
 \caption{Redis commands affected by parameter overwrite injection}
 \label{tab:redis_commands_affected}
\end{table}

The investigation revealed, that many of the commands allow parameter overwrites and therefore corruption on of the underlying data. When the first key argument is controlled by an attacker, all following can be overwritten. These vulnerabilities enable an attacker to perform unintended insert, update or delete operations.

\section{CouchDB}
Within this section, the discovered injection attacks against the document store CouchDB are outlined.

\subsection{Find Selector Injection}
This attack is based on the common attacker model and directed against the find operation of CouchDB. The database facilitates two techniques to retrieve data. One is to reference the required document by its key. When selection criteria for other properties is needed, documents are queried with JSON-like selection criteria and the special \emph{\_find} document. This document is available for every database collection and resolves the passed selection criteria. The find selector injection focuses on manipulation of the provided selection criteria. Similar to MonogDB, complex criteria, like \emph{greater than} and \emph{not equals}, is encoded by query selector objects. These can be misused by object structure injection. The example given by listing \ref{lst:FindSelectorInjectionNodeJS} implements an credentials check and is vulnerable for find selector injection. \\

\begin{lstlisting}[caption={Vulnerable NodeJS example for find selector injection on CouchDB}, label={lst:FindSelectorInjectionNodeJS}, language=JavaScript]
function checkCredentials(user, password, callback) {
  var options = {'selector': {'user': user, 'password': password}};
  couch.use('users').get('_find', options, (err, res) => {
    callback(res.docs.length === 1);
  });
}

checkCredentials(req.query.user, req.query.password, handleResult);
\end{lstlisting}

A function is implemented, that returns true or false depending on the success of the credentials check. The provided user and password value become part of the selector object, that is in turn contained within the options. Couch selects the user collection and performs a get operation with the help of the \emph{\_find} document and the composed options. Only when exactly one document is found that matches the selection criteria, the callback returns true. The described function is called with values provided by the request. Through object structure injection, the attack shown in listing \ref{lst:FindSelectorInjectionAttack} can be conducted. \\

\begin{lstlisting}[caption={Attack vector on CouchDB for speical key injection via HTTP GET}, label={lst:FindSelectorInjectionAttack}]
https://example.org/login?user=patrick&password[%24ne]=1
\end{lstlisting}

The payload contained within the query string of the URL leads to an object instead of a string typed password property. This object will replace the provided password value with a \emph{not equals} selector object within the query options. Therefore, the password check will reliably evaluate to true and the credentials check is bypassed. \\

\begin{table}[h]
 \sffamily
 \centering
 \begin{tabular}{ll}
  \textbf{Operator} & \textbf{Description} \\ \hline
  \$lt      & less than argument\\
  \$lte     & less than or equal to argument \\
  \$eq      & equal to argument \\
  \$ne      & not equal to argument \\
  \$gte     & greater than or equal to the argument \\
  \$gt      & greater than argument \\
  \$exists  & field exists \\
  \$type    & type equals \\
  \$in      & contained in list \\
  \$nin     & not contained in list \\
  \$size    & length equal \\
  \$mod     & modulus equals remainder \\
  \$regex   & regular expression \\
  \$and     & all contained selectors match \\
  \$or      & any of the contained selectors matches \\
  \$not     & selector does not match \\
  \$nor     & no contained selector matches \\
  \$all     & contains all elements of argument list \\ 
  \$elemMatch & at least one of the contained selectors matches \\
  \bottomrule 
 \end{tabular}
 \caption{CouchDB's selector operations utilized for find selector injection}
 \label{tab:couchdb_affected_selectors}
\end{table}

\subsection{Special Key Injection}
The special key injection against CouchDB rests upon the common attacker model. In order to understand this attack, it is vital to be aware of CouchDB' storage concept. All data needed for its operation is treated as a document. This concerns application data as well as meta data of the existing collections. Security settings, operation history and data views are stored as documents within each collection. The only difference in comparison to normal documents is the prefixed underscore of the id. Meta data documents are accessible for every user with reading access on the collection. Special documents like those can be used to bypass application layer checks. Listing \ref{lst:SpecialKeyInjectionNodeJS} displays an authentication check vulnerable to special key injection. \\

\begin{lstlisting}[caption={Vulnerable NodeJS example for special key injection on CouchDB}, label={lst:SpecialKeyInjectionNodeJS}, language=JavaScript]
function checkCredentials(user, password, callback) {
  couch.use('users').get(user, (err, res)=> {
    callback(res.password === password);
  });
}

checkCredentials(req.query.user, req.query.password, handleResult);
\end{lstlisting}

The snippet implements a \emph{checkUser} function, that receives a user and password argument. Since CouchDB's concept does not feature selection criteria, documents are retrieved by their id. Other properties of the stored documents, such as the password, cannot be filtered directly. In the presented case, the user property represents the document key and is used to query the stored record from the database. Password matching between the retrieved document and the passed argument is accomplished by the application layer. The result of the password matching is returned by the callback. \\

An attacker is able to pass this authentication check with the help of the stored meta data documents. The request presented in listing \ref{lst:SpecialKeyInjectionAttack} leads to a successful authentication.
 
\begin{lstlisting}[caption={Attack vector on CouchDB for speical key injection via HTTP GET}, label={lst:SpecialKeyInjectionAttack}]
https://example.org/login?user=_all_docs
\end{lstlisting}

With the provided user value, a special document is requested instead of of normal record. The database returns the \emph{\_all\_docs} document, that contains meta information about the collection. Important at that point is, that the special document does not contain a password property and is therefore undefined. Since the attack vector does not define a password value, the resulting argument is also undefined. Thus, two undefined values are compared for equality. JavaScript as well as Python, Ruby and PHP evaluate such a comparison to true and pass it to the callback. As a result, the credentials check is successfully bypassed.\\

CouchDB implements a group of these special documents for different purposes. Not all of these documents have the same access authorization level. Each of the special documents can be accessed by designated HTTP methods, as shown for a subset by table \ref{tab:couch_special_documents}. \\

\begin{table}[h]
 \sffamily
 \centering
 \begin{tabular}{llcccc}
  \textbf{Key} & \textbf{Description} & \textbf{GET} & \textbf{POST} & \textbf{PUT} & \textbf{DEL} \\ \hline
  \rowcolor{light-gray}\_all\_docs             & Return multiple documents of a database   & \cmark & \cmark & \xmark & \xmark \\
  \_bulk\_docs            & Create and update multiple documents      & \xmark & \cmark & \xmark & \xmark \\
  \rowcolor{light-gray}\_design                & Create or return views of database        & \cmark & \cmark & \xmark & \xmark \\
  \_find                  & Find documents with selection criteria    & \xmark & \cmark & \xmark & \xmark \\
  \rowcolor{light-gray}\_index                 & Get or create index                       & \cmark & \cmark & \xmark & \cmark \\
  \_explain               & Explain database structure                & \xmark & \cmark & \xmark & \xmark \\
  \rowcolor{light-gray}\_changes               & List of changes in the database           & \cmark & \cmark & \xmark & \xmark \\
  \_compact               & Run compression of database               & \xmark & \cmark & \xmark & \xmark \\
  \_ensure\_full\_commit  & Commits changes to disk                   & \xmark & \cmark & \xmark & \xmark \\
  \rowcolor{light-gray}\_security              & Return or set security object of database & \cmark & \xmark & \cmark & \xmark \\
  \_purge                 & Remove references to deleted documents    & \xmark & \cmark & \xmark & \xmark \\
  \_missing\_revs         & Return not stored revisions               & \xmark & \cmark & \xmark & \xmark \\
  \_revs\_diff            & Returns diff to stored revisions          & \xmark & \cmark & \xmark & \xmark \\
  \rowcolor{light-gray}\_revs\_limit           & Returns current revision limit            & \cmark & \xmark & \cmark & \xmark \\
  \bottomrule 
 \end{tabular}
 \caption{CouchDB special documents with description and applicable HTTP methods}
 \label{tab:couch_special_documents}
\end{table}

Depending on the operational purpose of the special document, designated HTTP methods are enabled. This has to be considered for attacks, since the injected special document needs to support the applied HTTP method of the exploited query. In case of the shown injection example, the HTTP GET method had to be supported. A major advantage of an attacker is, that even the lowest authorization level is allowed to GET special documents. Special key injection against queries using other HTTP methods require a higher authorization level of the underlying database user. A correctly configured application server and database should prevent higher authorization levels for normal database operations. \\

\subsection{Array Value Injection}
This attack is based on the common attacker model and directed against preliminary data validation. CouchDB implements protective measures for its special documents, but user defined documents have to be protected otherwise. Data checks within the application layer represent a solution for data protection. Inputs leading to illegitimate operations can be filtered beforehand in order to prevent query execution. At this point it is essential to filter all inputs giving a rise to the protected operation. Reliable filtering is hard to guarantee, when conditionals are handled differently on the application and database layer. Listing \ref{lst:couch_array_key_injection} gives an example for document protection, that can be bypassed by array value injection.\\

\begin{lstlisting}[caption={Vulnerable NodeJS example for array key injection on CouchDB}, label={lst:couch_array_key_injection}, language=JavaScript]
function getDocument(key, callback) {
  if (key === "secretDoc" || key[0] === "_") {
    callback("No access!");
  } else {
    couch.use('documents').get(key, callback);
  }
}

getDocument(req.query.key, handleResult);
\end{lstlisting}

The given code implements a function to retrieve documents by key. In order to protect the access on special documents and the \emph{secretDoc}, filters are applied. Special keys are filtered by the the first underscore char and the secret document is protected by direct key comparison. Every other key is passed to the query and the according document is returned. An attack has to search for a second query syntax, that allows access to the protected documents and all the same passes the preliminary checks. Listing \ref{lst:couch_array_key_injection_attack} presents two requests, that bypass the protective measures and retrieve the documents. \\

\begin{lstlisting}[caption={Attack vectors on CouchDB for array key injection via HTTP GET}, label={lst:couch_array_key_injection_attack}]
https://example.org?key[]=secretDoc
https://example.org?key[]=_all_docs
\end{lstlisting}

Basically, the attacks wrap an array around the passed key value. This achieves a different treatment of the parameter in the application and database layer. The filters use type safe condition evaluation. Therefore, the first attack vector injects an array which is not equal to the \emph{secretDoc} string. The second attack vector bypasses check, since the first element of the array does not equal the underscore char. Within the database layer, arrays containing just one element are resolved to the contained value. In both cases, the sensitive documents are returned. For this kind of vulnerability, similarities between string and array objects in the NodeJS, PHP, Ruby and Python environment become critical.   

\subsection{URL Traversal Injection}
The URL traversal injection against CouchDB rests upon the common attacker model. This attack relies on CouchDb's REST interface, that is used for any communication between application and database layer. REST uses URLs to identify the targeted documents. Some query parameters, such as the key, are used to build these URLs. Based on this, the key has influence on the structure of the URL. Applications as shown in listing \ref{lst:couchdb_url_traversal_injection_app} are vulnerable for URL traversal injection. \\

\begin{lstlisting}[caption={Vulnerable NodeJS example for URL traversal injection on CouchDB}, label={lst:couchdb_url_traversal_injection_app}, language=JavaScript]
function getDocument(key, callback) {
  couch.use('table').get(key, callback);
}

getDocument(req.query.key, handleResult);
\end{lstlisting}

\begin{lstlisting}[caption={Attack vectors on CouchDB for URL traversal injection via HTTP GET}, label={lst:PHPArrayInjection}]
https://example.org/get?key=_design/credentials/_view/user_password
\end{lstlisting}


\subsection{Data Import Injection}

\section{Memcached}
This section covers the injection attacks against the key-value cache Memcached, that were found in scope of this thesis.
\subsection{Array Key Injection}
  \chapter{Conception}
\label{cha:conectpion}

\section{Paradim Shift for NoSQL Queries}

\section{Approaches for Injection Attacks}

\subsection{Altered Object Type Injection}

\subsection{Function Injection}
  \chapter{NoSQL Injection Mitigation}
Within this section, a mitigation approach for injection attack against non-relational databases is presented. Based on the previous attack analysis and classification, a new prevention concept is elaborated addressing the major design problems. Further, a feasible way for the implementation of the presented mitigation technique is outlined and empirically evaluated.

\section{Conception}
In order to create an mitigation concept for NoSQL injection, the exposed design problems have to be considered. The last chapter concludes four major issues, that lead to injection vulnerabilities of non-relational databases. Since error-prone string escaping is a well known and actually problem, the fourth class is not focused. The main issue brought along with non-relational databases is object structure defined semantic. This feature gives a raise to attacks based on error-prone type and structure escaping as well as diverging parameter handling. As exposed by the found attacks, problems of this kind are spread across all investigated databases and application layers. The third class of attacks is bases on shared storage scopes and is only present for CouchDB. This issue can be solved by strict data separation. Therefore, the main goal for attack mitigation is the prevention of type and object structure injection. This addresses two-thirds of the found injection attacks and represents a problem not adequately solved yet. \\

The first idea regarding the mitigation of type and object structure based injection attacks, may be simple type casting of parameters. Although, this strict method would prevent most attacks, it has major drawbacks. On the one hand, the required type casting is highly dependent on the use case and performed query. With this technique, developers are responsible for attack mitigation by applying suitable type castings for each query parameter. This resembles the idea of manual parameter escaping for SQL statements, which was not reliably applied by developers. A vital argument against type casting, is the flexibility required by many applications. Listing \ref{lst:ExampleFindQueryStringID} and \ref{lst:ExampleFindQuerySpecialID} exemplify a use case, that employs two different types of identifiers. \\

\begin{minipage}{.97\textwidth}
\begin{minipage}[t]{.49\textwidth}
\begin{lstlisting}[escapechar=!, caption={Example for find query with a string identifier}, label={lst:ExampleFindQueryStringID}]
db.find({
  "_id": !\textbf{"56767834"}!

});
\end{lstlisting}
\end{minipage}
\hfill
\begin{minipage}[t]{.49\textwidth}
\begin{lstlisting}[escapechar=!, caption={Example for find query with a special object identifier}, label={lst:ExampleFindQuerySpecialID}]
db.find({
  "_id": {
    !\textbf{"\$oid": "54651022bffebc03098b4567"}!
});
\end{lstlisting}
\end{minipage}
\end{minipage}

The given queries work on a database, that contains documents with varying identifier formats. Some documents use a string based identifier, others employ an object-based identifier. The latter type is used in order to indicate a special format of the stored identifier string. Such differences of data types are typical for non-relational databases, due to their ability to handle unstructured data. The highlighted parts of the code have to be under user control, to allow generic querying of both identifier types. In contrast to the given scenario, such querying becomes arbitrary complex depending on the number of different property types within the database. Type casting for the highlighted parameter can therefore not be applied. In summary, some use cases require structural control of user-provided query parameters. \\

So why not apply type casting of parameters where possible and otherwise grant full flexibility? Regarding the example given with listing \ref{lst:ExampleFindQueryStringID} and \ref{lst:ExampleFindQuerySpecialID}, query selector injection is still an issue. Therefore, the full flexibility has to be restricted a bit. At this point, the mitigation approach of this thesis applies. In order to grant a certain degree of flexibility, but still prevent injection, allowed parameter structures have to be defined for each query. Based on the requirements, this thesis suggest a pattern-based control mechanism for query parameterization. The patterns define a range of allowed query structures and the all others are blocked. This approach applies directly at the connection between application and database layer, within the database driver. Therefore, each input for the query can be reliably filtered without any danger of subsequent manipulation. \\

The presented approach still requires manual pattern definition for each query, but in contrast to type casting at least allows some kind of flexibility for query parameters. This idea can be taken a step further to bring another important advantage. The creation of the applied security patterns can be automatized. Basically, the idea is to learn the patterns in a secure execution environment of the application. In a second step, the application goes live and the pattern validation is activated. This can be seen as a learning and an execution phase. Software tests are normally present for each application and constitute a great means to build the security patterns based on trusted inputs. By providing such a framework, a solution independent from the underlying technology stack, concrete security patterns and implementation details is given. This approach provides the needed flexibility, does not require high engineering efforts and provides sufficient protection from the found injection attacks.

\section{Implementation}
The previously described mitigation technique for NoSQL injection attacks has to be transformed into practice. For the exemplary implementation outlined within this section, the most prevalent technology stack was selected. According to the statistics shown in chapter \ref{cha:intro_to_nosql_injection}, MongoDB and NodeJS represent the most widespread technologies relevant in scope of this thesis. This combination also features the most prevalent technology stack with a demand for flexible object structure sanitizing Therefore, the objective comprises the implementation of the presented mitigation technique as a MongoDB driver for NodeJS. \\ 

\begin{figure}[h]
\centering
  \includegraphics[width=1\linewidth]{Images/secure_driver}
  \caption{Architecture for the implementation of the NoSQL injection mitigation concept}
  \label{fig:architecture_secure_driver}
\end{figure}


\begin{lstlisting}[escapechar=!, caption={Proposed security pattern grammar for the the injection mitigation mechanism}, label={lst:http_request_example}]
<SecurityPattern> ::= {'_security_pattern_': <Option> };
<Options> ::= <Option> | <Option>, <Options>;
!\textbf{<Option> ::= [<Values>]}!;
<Value> ::= <Array> | <Object> | <Type>;
<Array> ::= [] | [!\textbf{<Options>}!];
<Object> ::= {} | {<Properties>};
<Values> ::= <Value> | <Value>, <Values>;
<Properties> ::= <Property> | <Property>, <Properties>;
<Property> ::= <Key> : !\textbf{<Option>}!;
<Type> ::= "String" | "Number";
<Key> ::= *Arbitrary String*;
\end{lstlisting}

\begin{lstlisting}[escapechar=!, caption={Security pattern allowing string-based and object-based identifiers as a query parameter}, label={lst:http_request_example}]
{"_security_pattern_": [{
  "_id": ["String", {"$oid": ["String"]}]
}]}
\end{lstlisting}

% selected technology stack nodeJS + MongoDB - most prevalent combination with demand for flexible object structure sanitizing
% take a look at current driver
% slelect affected query functions and parameters
% add optional security fearture for object strucutre snaitzing
% implemented as additonal parameter
% show new function arguments
% pattern needs to be flexible concerning data types
% exmaple could be a number or an array of nubmers
% extended JSON syntax for security pattern
% exmaple for security patterns


\section{Evaluation}
\label{sec:evaluation}

\subsection{Methodology}

\subsection{Compatibility}

\subsection{Security}

  \chapter{Evaluation}
\label{cha:Empirical Evaluation}
  \chapter{Related Work}

\textbf{No SQL, No Injection? Examining NoSQL Security}\cite{Ron:2015} \\
  \chapter{Conclusion}
This final chapter discusses the results of this thesis with regard to the initially set objectives. Afterwards, an outlook for further research in scope of NoSQL injection is given.

\section{Discussion}
Injection attacks against non-relational databases represent an serious issue. System landscapes become more an more heterogeneous with direct impact on security. Therefore, existing attacker models for injection attack have to be reconsidered. This work introduces two new attacker models next to the established one, known from SQL injection. These are required to deliver a clear definition for injection attacks against NoSQL databases. With the given definition and analysis of attacker models, the first objective of this thesis is covered. Based on this, the most prevalent technology stacks were selected for further investigation. The research in scope of this work revealed multiple new injection vectors across all selected databases and application layers. This proves, that injection attacks against non-relational databases are not individual cases, like previous publications indicated. With the broad spectrum of investigated technology stacks, an holistic view of the subject is given for the first time. The second objective of this work is therefore covered comprehensively. Likewise, the presented research broke down the found injection vectors and exposed multiple conceptional problems. This attack classification gives a clear overview of the faced challenges for NoSQL injection prevention. Especially, the new categories of object structure and type injections were not approached until now. The third objective set in scope of this work is therefore addressed. Ensuing from the attack classification and requirements of non-relational databases, a mitigation approach was presented, that considers flexibility, security and feasibility. As the empirical evaluation showed, the presented solution can be integrated into existing solution without any problems. The generic approach provides and additional layer of security and compatibility across all investigated platforms, as the practical implementation proves. Therewith, also the fourth and last objective was successfully covered. \\

All in all, NoSQl injection represent more serious problem than expected thus far. The variety of injection vectors and classes of attacks also requires new approaches for a secure operation of non-relational databases. 

\section{Outlook}

\end{onehalfspace}

% ---------------------------- Literaturverzeichnis ----------------------------------------------
\printbibliography

% ------------------------------- Anhang ---------------------------------------------------------
%\begin{appendix}
%\clearpage Appendix
%\end{appendix}

\end{document}