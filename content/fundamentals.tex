\chapter{Technical Background}
\label{cha:technicalBackground}
This chapter outlines the essential technologies and techniques that are relevant for the further investigation of NoSQL injection. Thereby, the underlying communication protocols, the different concepts of NoSQL databases as well as security aspects are covered.

\section{Underlying Technologies}
Within this section, the prevalent protocols and paradigms for data exchange between and within system are explained.

\subsection{HTTP}
The Hypertext Transfer Protocol (HTTP) represents an application-layer protocol for the exchange of inter-connected multimedia documents \cite{Berners-Lee1996}. Initially, HTTP was created regarding web-server to web-browser communication, but today the protocol is applied for various purposes \cite{Fielding:1999}. As a stateless protocol following a client-server model, HTTP itself does not retain any state between requests. While the protocol is often applied as a part of the TCP/IP stack, it is also applicable with other reliable and error-checked protocols on the transport layer. A required header block and an optional body block build the basic structure of each request and response. Depending on the intended kind of operation, different methods for requests exist. Table \ref{tab:http_methods_overview} gives an overview of the available HTTP methods and their characteristics \cite{Fielding:1999}.\\

\begin{table}[h]
 \sffamily
 \centering
 \begin{tabular}{llccc}
  \textbf{\begin{tabular}{@{}c@{}}HTTP \\ Method\end{tabular}} & \textbf{Description} & \textbf{\begin{tabular}{@{}c@{}}Request \\ Body\end{tabular}} & \textbf{\begin{tabular}{@{}c@{}}Response \\ Body\end{tabular}} & \textbf{Idempotent} \\ \hline
  GET     & Retrieve specified resource & \xmark & \cmark & \cmark \\
  HEAD    & Retrieve specified resource headers & \xmark & \xmark & \cmark \\
  POST    & Create or update specified resource & \cmark & \cmark & \xmark \\
  PATCH   & Partially update specified resource & \cmark & \cmark & \xmark \\
  PUT     & Create or replace specified resource & \cmark & \cmark & \cmark \\
  DELETE  & Delete specified resource & \xmark & \cmark & \cmark \\ \hdashline
  CONNECT & Convert connection to tunnel & \cmark & \cmark & \xmark \\
  OPTIONS & Return supported HTTP methods & \cmark & \cmark & \cmark \\
  TRACE   & Echo received request & \xmark & \cmark & \cmark \\
  \bottomrule
 \end{tabular}
 \caption{Overview of HTTP methods and their characteristics}
 \label{tab:http_methods_overview}
\end{table}

Basically, the first six methods implement create, read, update and delete operations for resources resp. documents on the server. Thereby, a URI in the request header specifies the resource, that the operation should be applied to. In case of create or update operations, the optional body block in the request contains the new resource data. Since resource retrieving and deleting methods do not require new data, they work solely with the request header block. All operations, apart from the HEAD request, deliver the result statement or document within the response's body block. The idempotent characteristic indicates, whether the behaviour of a method stays the same for multiple consecutive requests. Especially update operations result in idempotent behaviour. The last three methods represent helping functions for connection inspection and TCP tunnelling. An example for a request, that retrieves a specified resource form the server, is given in listing \ref{lst:http_request_example_HTTP_GET} \cite{MDN:2016}. \\

\begin{lstlisting}[caption={Example for HTTP GET request}, label={lst:http_request_example_HTTP_GET}]
GET /index.html HTTP/1.1
Host: example.org
Accept-Language: en
\end{lstlisting}

The first line of the request header is composed of the HTTP method, followed by a resource identifier and completed with the version of the used protocol. In the given example a GET request is used in order to retrieve the \emph{index.html} file. Afterwards, the host of the server is defined by the domain name. Accepted types of files, char-sets, encodings, languages and many more settings can be defined as additional headers. In case the requested resource exists and no error appears during request processing, the server sends a response as exemplified in listing \ref{lst:http_response_example_HTTP_GET} \cite{MDN:2016}. \\

\begin{lstlisting}[caption={Example for HTTP GET response}, label={lst:http_response_example_HTTP_GET}]
HTTP/1.1 200 OK
Date: Mon, 07 Nov 2016 09:30:00 GMT
Server: Apache
Last-Modified: Wed, 02 Nov 2016 20:18:22 GMT
Accept-Ranges: bytes
Content-Length: 29769
Content-Type: text/html

<!DOCTYPE html... (here come the 29769 bytes of the requested resource)
\end{lstlisting}

In the first line of the response, the server version of the protocol as well as the status and message code are returned. Depending on the success or failure of the request, the status code and according message in the response vary. Further headers contain meta information about the server, the requested resource and the appended body block. A blank line separates the body block from the preceding header. In the given example, the body contains the initially requested HTML document. With the end of the transferred response body, the HTTP request-response cycle is finished. \\

There exist various formats for the data contained within the body. The \emph{Content-Type} header states the type of format in order to guarantee correct handling of the appended data. Values of this header follow the Multipurpose Internet Mail Extension (MIME) standard. These are divided into two level declarations and encompass all major data formats. Examples for the widespread types used in the context of HTTP are \emph{text/html}, \emph{application/javascript} or \emph{text/css}.

\subsection{REST}
Representational State Transfer (REST) describes a programming paradigm for distributed systems in a stateless client-server environment \cite{Fielding:2000}. It specifies an architectural style, that enforces a consistent system interface design. Therefore, REST represents an alternative to prior approaches, such as SOAP or WDSL. The paradigm separates resource location and the applied method. Functional instructions are therefore not allowed to be included within the URI. A special implementation or protocol is not defined by the paradigm, whereby REST is commonly used in combination with HTTP and HTTPS. In the course of this, the HTTP methods encode the functional instructions and resources have to be uniquely identified by the requested URI. Despite strong conceptual similarities, the usage of HTTP and HTTPS does not imply REST conformance. Methods are often misused for different functionalities and as a result REST conformance is not provided.

\section{NoSQL Databases}
This section gives an introduction to NoSQL databases and the idea behind this new class of data storage. NoSQL databases and their concepts, that are relevant in scope of this work, are outlined in detail.

\subsection{Overview}
In 1998 the word \textit{NoSQL} was first used by Carlo Strozzi in context of a relational, document-oriented database designed to be accessed without SQL \cite{Strozzi:2007}. More than 10 years later in 2009, Johan Oskarsson reintroduced the term \textit{NoSQL} in oder to find a collective name for the increasing number of distributed, non-relational databases \cite{Oskarsson:2009}. From this time forward, NoSQL was used with the meaning of \textit{not only SQL} or \textit{non-relational} instead of \textit{no SQL}. Therefore, the expression builds an umbrella term for databases without relational restrictions and in many cases missing ACID support. The idea for such databases arose back in the 1960s \cite{IBM:2016, Nelson:1965}, but the actual trend started with the challenges of big data. NoSQL databases addressed these challenges through their flexible data models, distributed storage, horizontal scaling abilities and independence from dedicated hardware. Currently around 225 databases are prevalent, that denote themselves as NoSQL storage \cite{Edlich:2016}. Despite the shared non-relational concept, NoSQL databases can be divided in different categories. Depending on the focused field of application, there exist four major groups:

\begin{description}
\item [Key-value Stores] These databases reference datasets with unique keys. Datasets are accessed with the key, that retrieves a pointer to the dataset from a hash table. Therefore, data access over the key is very fast, but querying or updating single properties of a dataset is inefficient. Notable examples for key-value stores are Redis, Riak, Voledmort or Memcached.
\item [Document Stores] The main idea here is similar to key-value stores, but datasets also are composed of semistructured key-value pairs. These datasets are often called documents and are persisted in formats like JSON. This allows more efficient querying for single properties of a dataset. Notable examples for document stores are MongoDB, CouchDB or CouchBase.
\item [Graph databases] This group uses a flexible structure with nodes and edges to model graph structures. Depending on the implementation and the underlying data persistence, the data access differs. Some databases implement RESTful interfaces, others provide dedicated drivers for querying. Notable examples for graph databases are Neo4j, Titan or Graphite.
\item [Column Stores] Similar to relational databases, data is very structured, but the underlying persistence layer differs strongly. Data is arranged column after column instead of row after row. This column oriented storage allows the processing and especially analysis of large amounts of data over distributed systems. Notable examples for column stores are Apache Cassandra, HBase or SAP HANA.
\end{description}

Apart from the presented categories of NoSQL databases, there also exist hybrid models implementing features originating from multiple categories. An example is OrientDB, that combines a document store with graph database abilities. 

\subsection{MongoDB}
The non-relational, document-oriented database MongoDB was published as an open-source project in 2009 \cite{Eliot:2010}. Deriving its name from the word humongous, it is designed to provide high performance, high availability as well as automatic horizontal scaling \cite{MongoDB_Intro:2016}. The JSON style data model enables complex hierarchies for stored documents, but still allows efficient querying and indexing. Documents are stored in binary JSON (BSON) format and separated by collections. These collections represent the unstructured counterpart to relational tables. The querying of stored documents is accomplished with the help of JSON expressions. BSON documents and JSON queries are compatible, since the former format is just an binary encoded extension of JSON with additional data types, such as date, time stamps and regular expressions. By default, every MongoDB instance holds special system collections for indexes, namespaces, users and JavaScript code. Application and database meta data are clearly separated by collections. MongoDB provides official drivers for various popular programming languages next to many unofficial community-supported drivers. In addition to these drivers, also a RESTful interface is provided.

\subsection{Redis}
Redis is an in-memory operating key-value store for strings, hashes, lists and sets \cite{Sanfilippo:2016}. The 2009 released open-source database allows range queries and has built in Lua support for scripting. Due to its low data structure complexity, Redis provides large performance advantages in comparison to relational databases. Snapshotting or append only files allow on-disk persistence of the in-memory stored data. Later versions also allow distributed storage across clusters. Redis provides driver bindings for the most prevalent programming languages.

\subsection{CouchDB}
CouchDB is a document-oriented database written in Erlang \cite{Anderson:2010}. First published in 2005, the database is now maintained by the Apache Software Foundation and distributed under the Apache license. CouchDB is designed to combine the relaxed data model of document stores as well as the performance and scalability of relational databases. The underlying storage structure for documents resembles the JSON format. Documents are commonly accessed by their key and predefined views, but JSON-like queries for properties are also possible. Therefore, the database offers horizontal scaling and implements the MapReduce framework for processing of distributed data. On Account of this, CouchDB includes Mozilla's JavaScript engine SpiderMonkey to facilitate scripting. Similar to other document stores, CouchDB employs the concept of collections for data separation. Meta documents are used in order to define views, access restrictions or general settings. These are stored within the according collections and require a prefixed key. Authorization for data access is divided in three roles. CochDB admins have overall control, database admins have complete control over single collections and database users can read all documents within a single collection. Reading access also includes the meta data documents and is based on GET request. Access for all major programming languages is provided with a RESTful HTTP interface, but dedicated drivers are also available.

\subsection{Memcached}
The in 2003 released database Memcached is an in-memory key-value store for data caching \cite{Dormando:2015}. It allows to store and retrieve data from memory with hash tables distributed across multiple systems. Therefore, the result of requests to external sources can be cached and accessed much faster for subsequent requests. Since the database is designed as a cache, data persistence on disc is not provided by default. Due to the resulting performance benefits of this approach, Memcached achieved wide prevalence and is deployed by companies like Facebook, YouTube, Twitter and Wikipedia. As typical for key-values stores, the interface is kept straightforward and focuses on get and set operations. Drivers are provided for all major programming languages.

\section{Application Security}
In this section an outline of the relevant security topics in the context of this thesis is given. Therefore, general security goals are introduced, the current state of NoSQL security is blueprinted and afterwards, the concept of injection attacks is described.

\subsection{Security Goals}
Information security is a general term comprising the practice of data protection by defending it against attacks \cite{olivier2002database,Perrin:2008}. It involves three major security goals, that have to be assured for the underlying data. These goals describe a security model, known as the CIA triad. This fundamental concept of information security is defined as follows:

\begin{description}
\item [Confidentiality] Data has to be protected against any kind of unauthorized disclosure and the authenticity of data sources has to be ensured.
\item [Integrity] Data hast to be protected from any kind of unauthorized modification and its consistency, accuracy as well as trustworthy have to be ensured.
\item [Availability] Data hast to be accessible for authorized entities and usability as well as timely access have to be ensured. A delay or denial of access for authorized entities as well as unauthorized removal of data have to be prevented.
\end{description}

Vulnerabilities like missing encryption, software bugs, weak passwords or inaccurate input validation enable an attacker to break one of these goals for the targeted data. Attacks in turn are the act of exploiting vulnerabilities. Most security measures are designed in order to prevent suchlike attacks.

\subsection{NoSQL Security}
Non-relational databases represent a relatively new technological approach, thus security aspects did not attain perfection by now \cite{Okman:2011, Noiumkar:2014}. In order to give an overview of these security aspects, multiple facets of the current state of NoSQL security have to be considered. Non-volatile data persistence is an essential subject for databases. The data persisted on disc has to be protected from illegitimate disclosure from other entities. Unfortunately, many non-relational databases store data unencrypted. This situation leverages direct data access from disc, when the underlying system or one of its applications is compromised. Other vital features of a database are authentication and authorization checks. By default, instances of many NoSQL databases run without these mechanisms. This circumstance allows uncontrolled access on the stored data. When the database ports are exposed to the outside, attackers are able to run remote attacks. Closing these ports to the outside still leaves room for CSRF attacks, thanks to HTTP-based interfaces. Distributed storage is an major concern for non-relational databases and therefore inter-cluster communication becomes necessary. Some of the new databases are missing encryption for connections between distributed instances, enabling eavesdropping of transferred data. Last of all, also injection attacks represent an important security aspect. Despite other query languages than SQL are used, the concatenation of string parameters still reveals attack surface. However, injection attacks against NoSQL databases constitute a relatively new and unexplored issue entailing a broad spectrum for detailed research.

\subsection{Injection Attacks}
The class of injection attacks encompasses a variety of actual attacks directed against specific contexts \cite{OWASP:2013}. These types of attacks are ranked as the number one web application security risk according to the OWASP top ten project \cite{OWASP:2013b}. An injection vulnerability allows manipulation of later-on interpreted contents. Suchlike vulnerabilities occur, when data originating from untrusted sources is passed non-sanitized to an interpreter. Untrusted sources are user controlled and therefore offer attack surface for manipulation. In order to conduct an injection attack, the attacker crafts a text-based payload and passes it to the system. This payload exploits the syntax of the targeted interpreter to achieve unintended behaviour. As a result, manipulated contents originating from the untrusted source are interpreted and executed. Depending on the attacked interpretation context, such attacks lead to data corruption, data loss, denial of access or even hostile system takeover. Common contexts targeted by injection attacks are SQL, Shell, LDAP or XML. Best practice for injection mitigation is the deployment of parameterized interfaces or alternatively stored procedures. The OWASP also recommends strict input validation by white-listing in feasible cases.
