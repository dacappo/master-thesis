\chapter{NoSQL Injection Mitigation}
Within this section, a mitigation approach for injection attack against non-relational databases is presented. Based on the previous attack analysis and classification, a new prevention concept is elaborated addressing the major design problems. Further, a feasible way for the implementation of the presented mitigation technique is outlined and empirically evaluated.

\section{Conception}

% first select the underlying problem
% string escaping already addressed - soltion is available, just not correctly applied
% object structure escaping instead not covered by any mitigation technique
% casting is not a soltion for flexible but restricted cases
% need a pattern for security senstive data calls
% taint tracking not a solution, since some structural elmenets can be in user control

% so far the pattern has to be set by the developer
% this can be automatized
% idea of lerning phase in secure environment and live execution phase
% pattern can still be adapted manually
% seucre environment could be tests
% gives all the advantages and is secure, when patterns are sufficiently learned

\section{Implementation}

% selected technology stack nodeJS + MongoDB - most prevalent combination with demand for flexible object structure sanitizing
% take a look at current driver
% slelect affected query functions and parameters
% add optional security fearture for object strucutre snaitzing
% implemented as additonal parameter
% show new function arguments
% pattern needs to be flexible concerning data types
% exmaple could be a number or an array of nubmers
% extended JSON syntax for security pattern
% exmaple for security patterns


\section{Evaluation}
\label{sec:evaluation}

\subsection{Methodology}

\subsection{Compatibility}

\subsection{Security}
