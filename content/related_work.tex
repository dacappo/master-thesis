\chapter{Related Work}
This chapter outlines related work and publications regarding NoSQL Injection attacks and prevention. First an overview of known NoSQL injection vectors is given and afterwards existing approaches for injection prevention are summarized. Thereby similarities as well as the major differences in comparison to this thesis are discussed.

\section{NoSQL Injection}
Within this section, the publications of known NoSQL injection vectors are summarized in chronological order. Afterwards a comparison to the results of this thesis is given. \\

\textbf{Security - NoSQL-Injection}\cite{Oftedal:2010} \\
The presence of injection attacks on NoSQL databases was first demonstrated in a blog post by Erlend Oftedal in 2010 \cite{Oftedal:2010}. He described attacks on MongoDB based on string concatenation of parameters. Thereby BSON objects or JavaScript expressions are build with user controlled data and then passed to the database. In the former case the attacker is able to alter the BSON object structure and therefore the semantics of the query parameter. In the latter case scripts, that are executed within the database, can be manipulated by an attacker. The behavior of queries can be influenced in either case. \\

\textbf{Sever-side JavaScript Injection}\cite{Sullivan:2011} \\
In 2011 Brian Sullivan presented further NoSQL injection vectors as a part of a BlackHat session. Next to CSRF attacks on NoSQL databases and NodeJS security issues, he illustrated two attack vectors for NoSQL injections based on MongoDB and PHP. Thereby the structure of parameters is altered through a PHP feature called associative arrays. These allow a special syntax for URL-based query strings, that are then interpreted as arrays in PHP. As soon as query string parameters are used as a part of database query parameters, arbitrary objects can be injected instead of string values. In combination with MongoDB's special keys, affected query criteria can be semantically manipulated. The second presented attack vector is executed in the same manner, but the special array key \textit{\$where} is used. As defined by the specification, MongoDB evaluates the value of this key as a script. Hence, a script injection for single query criterion can be performed. \\

\textbf{A Comparison the Level of Security on Top 5 Open Source NoSQL Databases} \cite{Noiumkar:2014} \\
A general overview of the current state of NoSQL security was given by Noikumar et al. in 2014. They examined the most popular NoSQL databases - MonogDB, Cassandra, CouchDB, Hypertable and Redis - with regard to encryption, authentication, script injection and DoS attacks. MongoDB and CouchDB exposed possible script injections, whereas the other databases where classified as secure for this kind of attack. Other types of injection attacks, as presented within this work, were not considered. \\

\textbf{Hacking Node.js and MongoDB}\cite{Petkov:2014a, Petkov:2014b} \\
Injection attacks for MongoDB in combination with NodeJS were published by Petko Petkov on his web security blog in 2014 \cite{Petkov:2014a, Petkov:2014b}. With one example he illustrated, how to bypass a typical login implementation. Similar to the known PHP associative array injection, some major modules of NodeJS allow a special syntax to insert arrays from the URL's query string. This in turn allows to alter query criteria by object injection. In the login case, a greater than comparison in contrast to the default equals comparison is inserted for the user and password value. As a result, both criteria are evaluated to true and the login is bypassed. In addition to the GET request based example, also a POST request version of his attack vector is shown. Within the second article, Petkov gave an example for his attack vector in realistic implementations. By use of object injections, a brute-force attack for hashed passwords on all users is possible.\\

\textbf{The New Page of Injections Book: Memcached Injections} \cite{Novikov:2014} \\
Security researcher Ivan Novikov presented a technique for command injection attacks on Memcached on BlackHat in 2014. Thereby he showed how to append additional commands to existing queries. By the injection of special bytes such as newline characters, the actual command is terminated. Since these special bytes were not escaped, the database interpreted trailing characters as a new command. Multiple versions of this attack for Ruby, Python and PHP with corresponding database drivers were revealed by the publication. The injection vector is not based on a conceptional problem of the database, but frequent implementation flaws in the database drivers. \\

\textbf{No SQL, No Injection? Examining NoSQL Security} \cite{Ron:2015} \\
One of the latest publications covering NoSQL injection was release by Ron et al. in 2015. The paper gives an overview of NoSQL injection vulnerabilities across databases. Known injection vectors such as the associative array injection on MongoDB via PHP \cite{Sullivan:2011} and string concatenation of objects \cite{Oftedal:2010} were resumed. Additionally, the researchers published an example for a MapReduce injection on MongoDB, that allows to execute scripts in scope of the database. This new attack vector works across application platforms and is based on concatenation of strings in order to build a map or a reduce function. Although, MapReduce is called for specific collections of MongoDB, in this scenario the attacker is able to access and change data of other collections. Proposed mitigation approaches were the encoding of input data and dynamic application security testing.\\

As former publications show, NoSQL injection is also an existing issue in the new generation of databases. In the majority of cases, only single attack vectors for specific database and application platform combinations are presented. MongoDB is often focused by security research, since it is the most prevalent database. By contrast, other NoSQL databases are hardly considered and only few injection vectors are documented. With respect to the application layer, that is an important factor for attacks, usually only PHP and NodeJS are taken into account. This thesis addresses multiple prevalent NoSQL databases in combination with the most common application layers and drivers. Therefore, a comprehensive overview of NoSQL injection can be given. In contrast to former publications concentrating on single attacks, this thesis reveals NoSQL injections as an overall issue across platforms.\\

\section{NoSQL Injection Prevention}
Within this section, the publications covering approaches for NoSQL injection mitigation are summarized and afterwards compared to the solutions proposed by this work. \\

\textbf{Diglossia: detecting code injection attacks with precision and efficiency} \cite{Son:2013} \\
The 2013 published paper by Son et al. suggest a system for precise injection detection for SQL as well as NoSQL databases. They propose a dynamic analysis of user controlled data and showcase the idea with an adapted PHP interpreter and extended database parser. A taint-tracking system marks user and system controlled characters of strings. For database calls, a special parser checks the taint information of all input parameters. In case taint is contained in sensitive tokens, the query is terminated. Otherwise, the normal parser of the database is called. Attacks on the structure of query parameters can be reliably prevented with this approach.\\

\textbf{Analysis and Mitigation of NoSQL Injections} \cite{Ron:2016} \\