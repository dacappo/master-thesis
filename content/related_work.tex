\chapter{Related Work}
This chapter outlines related work and publications regarding NoSQL Injection attacks and prevention. First an overview of known NoSQL injection vectors is given and afterwards existing approaches for injection prevention are summarized. Thereby similarities as well as the major differences in comparison to this thesis are discussed.

\section{NoSQL Injection}
Within this section, the publications of known NoSQL injection vectors are summarized in chronological order. \\

\textbf{Security - NoSQL-Injection}\cite{Oftedal:2010} \\
The presence of injection attacks on NoSQL databases was first demonstrated in a blog post by Erlend Oftedal in 2010 \cite{Oftedal:2010}. He described attacks on MongoDB based on string concatenation of parameters. Thereby BSON objects or JavaScript expressions are build with user controlled data and then passed to the database. In the former case the attacker is able to alter the BSON object structure and therefore the semantics of the query parameter. In the latter case scripts, that are executed within the database, can be manipulated by an attacker. The behavior of queries can be influenced in either case. \\

\textbf{Sever-side JavaScript Injection}\cite{Sullivan:2011} \\
In 2011 Brian Sullivan presented further NoSQL injection vectors as a part of a BlackHat session. Next to CSRF attacks on NoSQL databases and NodeJS security issues, he illustrated two attack vectors for NoSQL injections based on MongoDB and PHP. Thereby the structure of parameters is altered through a PHP feature called associative arrays. These allow a special syntax for URL-based query strings, that are then interpreted as arrays in PHP. As soon as query string parameters are used as a part of database query parameters, arbitrary objects can be injected instead of string values. In combination with MongoDB's special keys, affected query criteria can be semantically manipulated. The second presented attack vector is executed in the same manner, but the special array key \textit{\$where} is used. As defined by the interface, MongoDB evaluates the value of this key as a script. Hence, a script injection for single query criterion can be performed. \\

\textbf{A Comparison the Level of Security on Top 5 Open Source NoSQL Databases} \cite{Noiumkar:2014} \\
A general overview of the current state of NoSQL security was given by Noikumar et. al. in 2014. They examined the most popular NoSQL databases - MonogDB, Cassandra, CouchDB, Hypertable and Redis - with regard to encryption, authentication, script injection and DoS attacks. MongoDB and CouchDB exposed possible script injections, whereas the other databases where classified as secure for these kind of attack. Other types of injection attacks, as presented within this work, were not considered. \\

\textbf{Hacking Node.js and MongoDB}\cite{Petkov:2014a, Petkov:2014b} \\
Injection attacks for MongoDB in combination with NodeJS were published by Petko Petkov on his web security blog in 2014 \cite{Petkov:2014a, Petkov:2014b}. With one example he illustrated, how to bypass a typical login implementation. Similar to the known PHP associative array injection, some major modules of NodeJS allow a special syntax to insert arrays from the URL's query string. This in turn allows to alter query criteria by object injection. In the login case, a greater than comparison in contrast to the default equals comparison is inserted for the user and password value. As a result, both criteria are evaluated to true and the login is bypassed. In addition to the GET request based example, also a POST request version of his attack vector is shown. Within the second article, Petkov gave an example for his attack vector in realistic implementations. By use of object injections, a brute-force attack for hashed passwords on all users is possible.\\



\textbf{The New Page of Injections Book: Memcached Injections} \cite{Novikov:2014} \\

\textbf{No SQL, No Injection? Examining NoSQL Security} \cite{Ron:2015} \\


\section{NoSQL Injection Prevention}

\textbf{Diglossia: detecting code injection attacks with precision and efficiency} \cite{Son:2013} \\

\textbf{Analysis and Mitigation of NoSQL Injections} \cite{Ron:2016} \\