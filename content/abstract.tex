\chapter*{Abstract} %*-mark to no appear in the contents listing
In the last decade many new challenges, such as big data, changed the way we build applications. The generation of emerging NoSQL databases provides a solution for these challenges, but do they provide security? Regarding injection, there exists a prevalent opinion: "We are not building queries from strings, so we do not have to worry about injection vulnerabilities!" \cite{MongoDB_Fundamentals:2016} \\

NoSQL databases keep spreading into a wide range of modern applications, but security aspects stay behind. The fact that the most serious class of web attacks - injections - are not taken seriously in the context of NoSQL, calls for action \cite{OWASP:2013b}. This thesis addresses this problem and gives a comprehensive insight into the state of injection attacks against non-relational databases. Existing attack models have to be reconsidered due to the trend towards heterogeneous system landscapes. This work introduces two new attacker models emerging with NoSQL architectures. The most prevalent technology stacks including typical application and database layers are precisely examined for potential injection vulnerabilities. A detailed analysis of employed technologies reveals new ways to influence the semantics of database queries. The presented attacks are accompanied by multiple practical demonstrations. In a further step, the found attacks are broken down to the underlying conceptual problems. This attack classification facilitates the creation of suitable counter-measures. Beyond the described investigation, an approach for efficient NoSQL injection mitigation is introduced. The presented solution adds a layer of security by optional query parameter validation. With this approach, the flexibility required by non-relational data models is addressed and a seamless integration into to all investigated application platforms is achievable. As a whole, this thesis provides a deep insight into the subject of NoSQL injection and feasible mitigation approaches.