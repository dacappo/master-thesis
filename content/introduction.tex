\chapter{Introduction}
\label{cha:introduction}
This first chapter outlines the underlying motivation for the investigation of NoSQL injection, defines the objectives in scope of this work and gives a look forward to the achieved academic contributions. Lastly, the structure of the thesis is elucidated.

\section{Motivation}
In the last decade many new challenges, such as big data, social media and the internet of things, changed the way we build applications. More and more devices got connected to the internet and started producing huge amounts of unstructured data. This new situation demanded for highly scalable, always available systems for large numbers of concurrent and globally distributed users. The by then prevailing relational databases did not meet the upcoming requirements and reached their limits facing the enormous amounts of data. Given these circumstances, the generation of NoSQL databases emerged. With a non-relational approach in common, these databases provided a solution for the given challenges. Flexible handling of unstructured and semi-structured data originating from different sources in combination with powerful horizontal scaling capabilities lead to the breakthrough of NoSQL technologies. Since then, these databases were deployed in many application stacks and serve millions of users every day. Google uses NoSQL solutions as a part of their compute engine \cite{MongoDB_Google:2016}, Facebook adapted its storage engine for NoSQL integration \cite{MongoDB_Facebook:2016} and SAP selected NoSQL technologies for the core content management of its PaaS offering \cite{MongoDB_SAP:2016}. These and many more examples leave no doubt that this new generation of databases offers notable advantages for modern applications, but do they provide security? When innovative concepts find their way to production, the security aspect is often neglected. Especially a well-known attack vector against relational databases - SQL injection - was not taken into account for NoSQL applications. Regarding injection, the prevalent opinion is that we are not building queries from strings, so we do not have to worry about injection vulnerabilities. This attitude may rest upon the few known attack vectors for NoSQL injection against specific platforms. However, the launch of the social media platform Diaspora showed that NoSQL injection is a serious issue \cite{McKenzie:2010}. This raises the question, whether the known examples were individual cases or NoSQL injection presents a general threat for our modern applications. Since NoSQL comprises a variety of databases with slightly different concepts and query languages, a versatile attack surface for injection is given. All the same time, only little effort has been made to discover potential injection vulnerabilities. On account of this prevailing situation, the subject of NoSQL injection has to be investigated comprehensively and in greater detail.

\section{Objective}
\label{sec:objective}
This thesis aims to give an encompassing overview of injection attacks against NoSQL databases. A precise understanding of a system's architecture is vital to analyse potential risks. NoSQL injection presents a relatively new and unexplored kind of attack. Therefore, a clear definition of the class of attacks has to be made. Since NoSQL technologies lead to changes of typical system structures, also the existing attacker models have to be re-evaluated. The first objective of this thesis covers these subjects: \\

\textbf{Objective 1:} \textit{Detailed definition of NoSQL injection and the corresponding attacker models.} \\

Outgoing from these definitions concrete injection vulnerabilities in NoSQL databases have to be identified. Previous studies concentrated on single databases in combination with a specific application layer. In contrast to these published attack vectors, a more holistic approach is needed in order to give a widespread overview of the topic. On that account, the most prevalent NoSQL technology stacks have to be selected and investigated with regard to injection vulnerabilities. For this reason, the second objective of this work is defined as follows: \\

\textbf{Objective 2:} \textit{Investigation of a selected group of NoSQL technology stacks for injection vulnerabilities regarding the defined attacker models.} \\

With the results of the investigation, found vulnerabilities can be examined for their underlying problems. A differentiation between conceptual and implementation problems is an important aspect of this step. Vulnerabilities originating from implementation flaws can easily be fixed when reported, whereas conceptual problems are hard to solve in most cases. Conceptual vulnerabilities are based on the underlying design of the database. Therefore, the third objective covers the classification as follows: \\

\textbf{Objective 3:} \textit{Classification of known and found NoSQL injection vulnerabilities according to the underlying design problem.} \\

Based on the exposed design problems, the prevention of NoSQL injection attacks can be approached. An appropriate prevention mechanism should be able to reliably stop potential injection attacks, but all the same avoid false positives breaking existing applications. Thereby, a mitigation technique should seamlessly integrate into application stacks without requiring major code changes and engineering effort. The final objective of this thesis is therefore set in the following way:\\

\textbf{Objective 4:} \textit{Elaboration of an approach for NoSQL injection mitigation addressing the underlying design problems.} \\

The elaboration of these four objectives should allow a general overview of NoSQL injection and provide an outline of the threads modern applications have to face with this class of attacks. \\


\section{Contributions}

The main contributions of this thesis concerning NoSQL application security are:
\begin{description}
\item [Injection Attacks] NoSQL injection vulnerabilities are not a database or application layer specific problem. Therefore, this thesis reveals new injection vectors for multiple NoSQL databases in combination with common application layers. With this broad range of investigated technology stacks, it is shown that NoSQL injection is a general issue across non-relational databases.
\item [Attacker Models] Despite parallels to SQL injection attacks, the existing attacker models are not sufficient considering the NoSQL environment. Since NoSQL databases are created with the aim to handle huge amounts of unstructured data from several sources, format restrictions are low. These circumstances may allow direct or indirect insertions of attacker controlled data from external sources. On account of this, two additional attacker model are introduced to meet these new conditions.
\item [Injection Prevention] The mitigation of NoSQL injection is hard to ensure, since the data and therefore the requests assume various shapes across databases. In order to reduce the risk of injection attacks, this thesis breaks down known injection vectors to the underlying design problems. On basis of these, a new database driver design is presented, that is resistant to known injection attacks.
\end{description}

\section{Structure}
This thesis is structured as follows: In chapter two an overview of fundamental technologies of the thesis is given. Used protocols, investigated databases and the necessary security topics are outlined within this segment. Chapter three gives an introduction to the concept of NoSQL injection as well as corresponding attacker models. Chapter four approaches NoSQL injection in depth, examines the semantic structures of queries and analyses the given attack surface. Subsequently, the in scope of this thesis discovered injection vectors are presented in chapter five. The diversity of technology stacks is addressed within this segment. Chapter six classifies the found attack vectors with regard to their underlying problem. Based on this classification, chapter six presents a new approach for NoSQL injection mitigation. Within the same chapter, the presented solution is evaluated with real world applications. A comparison of this thesis to related publications is given in chapter eight. The last part, chapter nine, discusses the results of this thesis and gives a potential outlook for future work on the subject of NoSQL injection.