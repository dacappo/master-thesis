\chapter{Conclusion}
This final chapter discusses the results of this thesis with regard to the initially set objectives. Afterwards, an outlook for further research in scope of NoSQL injection is given.

\section{Discussion}
Injection attacks against non-relational databases represent an serious issue. System landscapes become more an more heterogeneous with direct impact on security. Therefore, existing attacker models for injection attack have to be reconsidered. This work introduces two new attacker models next to the established one, known from SQL injection. These are required to deliver a clear definition for injection attacks against NoSQL databases. With the given definition and analysis of attacker models, the first objective of this thesis is covered. Based on this, the most prevalent technology stacks were selected for further investigation. The research in scope of this work revealed multiple new injection vectors across all selected databases and application layers. This proves, that injection attacks against non-relational databases are not individual cases, like previous publications indicated. With the broad spectrum of investigated technology stacks, an holistic view of the subject is given for the first time. The second objective of this work is therefore covered comprehensively. Likewise, the presented research broke down the found injection vectors and exposed multiple conceptional problems. This attack classification gives a clear overview of the faced challenges for NoSQL injection prevention. Especially, the new categories of object structure and type injections were not approached until now. The third objective set in scope of this work is therefore addressed. Ensuing from the attack classification and requirements of non-relational databases, a mitigation approach was presented, that considers flexibility, security and feasibility. As the empirical evaluation showed, the presented solution can be integrated into existing solution without any problems. The generic approach provides and additional layer of security and compatibility across all investigated platforms, as the practical implementation proves. Therewith, also the fourth and last objective was successfully covered. \\

All in all, NoSQl injection represent more serious problem than expected thus far. The variety of injection vectors and classes of attacks also requires new approaches for a secure operation of non-relational databases. 

\section{Outlook}