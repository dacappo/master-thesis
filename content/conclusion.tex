\chapter{Conclusion}
This final chapter discusses the results of this thesis with regard to the initially set objectives. Afterwards, an outlook for further research in scope of NoSQL injection is given.

\section{Discussion}
Injection attacks against non-relational databases represent a serious issue. System landscapes become more and more heterogeneous with direct impact on security. Therefore, existing attacker models for injection attacks have to be reconsidered. This work introduces two new attacker models next to the established one, known from SQL injection. These are required to deliver a clear definition for injection attacks against NoSQL databases. With the given definition and analysis of attacker models, the first objective of this thesis is covered. Based on this, the most prevalent technology stacks were selected for further investigation. The research in scope of this work revealed multiple new injection vectors across all selected databases and application layers. Two of the three attacker models - the normal and the extended one - were affected by actual attack vectors, whereas the direct attacker model stayed unconcerned. This proves that injection attacks against non-relational databases are not individual cases, like previous publications indicated. With the broad spectrum of investigated technology stacks, a holistic view of the subject is given for the first time. Thus, the second objective of this work is covered comprehensively. Likewise, the presented research broke down the found injection vectors and exposed multiple conceptual problems. This attack classification gives a clear overview of the faced challenges for NoSQL injection prevention. Especially, the new categories of object structure and type injections were not approached until now. In this way, the third objective set in scope of this work is addressed. Ensuing from the attack classification and requirements of non-relational databases, a mitigation approach was presented that considers flexibility, security and feasibility. As the empirical evaluation showed, the presented solution can be integrated into existing implementations without any problems. The generic approach provides an additional layer of security and compatibility across all investigated platforms, as the practical implementation proves. Therewith, also the fourth and last objective was successfully addressed. \\

All in all, NoSQL injection represent a more serious problem than expected thus far. This work demonstrated that this class of attacks has to be considered with the entire technology stack in mind. The variety of injection vectors and classes of attacks also requires new approaches for a secure operation of non-relational databases. 

\section{Outlook}
NoSQL injection represents a relatively new class of attacks and offers a broad range for further research. This thesis examined the most relevant databases and application layers, but there exists a broad range of technology stacks not covered by investigation so far. Especially graph and column stores represent types of non-relational databases that are not addressed by research yet. An all-encompassing overview of the topic would be desirable. The immense amount of technology stacks constituted of application and database layers as well as applied frameworks and database drivers make an all-inclusive examination an ambitious challenge. Therefore, an important topic for further research is the classification of the underlying design problems. With the help of these classes, generic mitigation techniques against injection attacks can be developed and implemented across all platforms. The evaluated mitigation approach presented in scope of this thesis seems promising for the class of error-prone type checks. To deploy the idea in practice, further work on the security pattern learning and pattern matching has to be done. This is important to reduce false-positives in production and still guarantee the required flexibility. In general, the design of new approaches for injection mitigation constitutes a vital topic for further work. All in all, this thesis provides a holistic insight into the topic of NoSQL injection, but there is still a broad spectrum for the investigation of attacks and mitigation research. 